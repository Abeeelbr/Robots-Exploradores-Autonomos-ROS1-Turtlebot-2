%% Generated by Sphinx.
\def\sphinxdocclass{report}
\documentclass[a4paper,10pt,spanish]{sphinxmanual}
\ifdefined\pdfpxdimen
   \let\sphinxpxdimen\pdfpxdimen\else\newdimen\sphinxpxdimen
\fi \sphinxpxdimen=.75bp\relax
\ifdefined\pdfimageresolution
    \pdfimageresolution= \numexpr \dimexpr1in\relax/\sphinxpxdimen\relax
\fi
%% let collapsible pdf bookmarks panel have high depth per default
\PassOptionsToPackage{bookmarksdepth=5}{hyperref}

\PassOptionsToPackage{booktabs}{sphinx}
\PassOptionsToPackage{colorrows}{sphinx}

\PassOptionsToPackage{warn}{textcomp}
\usepackage[utf8]{inputenc}
\ifdefined\DeclareUnicodeCharacter
% support both utf8 and utf8x syntaxes
  \ifdefined\DeclareUnicodeCharacterAsOptional
    \def\sphinxDUC#1{\DeclareUnicodeCharacter{"#1}}
  \else
    \let\sphinxDUC\DeclareUnicodeCharacter
  \fi
  \sphinxDUC{00A0}{\nobreakspace}
  \sphinxDUC{2500}{\sphinxunichar{2500}}
  \sphinxDUC{2502}{\sphinxunichar{2502}}
  \sphinxDUC{2514}{\sphinxunichar{2514}}
  \sphinxDUC{251C}{\sphinxunichar{251C}}
  \sphinxDUC{2572}{\textbackslash}
\fi
\usepackage{cmap}
\usepackage[T1]{fontenc}
\usepackage{amsmath,amssymb,amstext}
\usepackage{babel}



\usepackage{tgtermes}
\usepackage{tgheros}
\renewcommand{\ttdefault}{txtt}



\usepackage[Sonny]{fncychap}
\ChNameVar{\Large\normalfont\sffamily}
\ChTitleVar{\Large\normalfont\sffamily}
\usepackage{sphinx}

\fvset{fontsize=auto}
\usepackage{geometry}


% Include hyperref last.
\usepackage{hyperref}
% Fix anchor placement for figures with captions.
\usepackage{hypcap}% it must be loaded after hyperref.
% Set up styles of URL: it should be placed after hyperref.
\urlstyle{same}

\addto\captionsspanish{\renewcommand{\contentsname}{Contenidos:}}

\usepackage{sphinxmessages}
\setcounter{tocdepth}{1}


        \usepackage{charter} % Cambia el tipo de letra
        \usepackage{amsmath} % Paquete para ecuaciones
        \setcounter{tocdepth}{2} % Ajusta la profundidad del índice
        \renewcommand{\familydefault}{\sfdefault} % Usa una fuente sans-serif por defecto
    

\title{Robots Exploradores - Práctica 3}
\date{24 de noviembre de 2024}
\release{1.0}
\author{Abel Belhaki Rivas}
\newcommand{\sphinxlogo}{\vbox{}}
\renewcommand{\releasename}{Versión}
\makeindex
\begin{document}

\ifdefined\shorthandoff
  \ifnum\catcode`\=\string=\active\shorthandoff{=}\fi
  \ifnum\catcode`\"=\active\shorthandoff{"}\fi
\fi

\pagestyle{empty}
\sphinxmaketitle
\pagestyle{plain}
\sphinxtableofcontents
\pagestyle{normal}
\phantomsection\label{\detokenize{index::doc}}


\sphinxAtStartPar
Bienvenido a la documentación del proyecto RobotsExploradores. Este proyecto está diseñado para trabajar con robótica móvil utilizando ROS y se divide en dos paquetes principales: \sphinxstylestrong{main} y \sphinxstylestrong{exploracion}.

\sphinxAtStartPar
Esta documentación detalla los componentes, configuraciones y pasos necesarios para poner en marcha este sistema robótico.


\chapter{Introducción}
\label{\detokenize{index:introduccion}}
\sphinxAtStartPar
El objetivo del proyecto es coordinar diversas tareas en un robot móvil. Estas incluyen detección de objetos, aproximación a puntos específicos y navegación autónoma. A continuación, se presentan los principales paquetes y componentes del sistema:

\sphinxAtStartPar
\sphinxstylestrong{Paquete Principal (main):}
\sphinxhyphen{} Implementa la máquina de estados para gestionar el flujo de trabajo del robot.
\sphinxhyphen{} Proporciona una interfaz gráfica para interactuar con el robot.

\sphinxAtStartPar
\sphinxstylestrong{Paquete de Exploración (exploracion):}
\sphinxhyphen{} Define los servidores de acción para tareas clave como detección de objetos y control autónomo.

\begin{sphinxadmonition}{note}{Nota:}
\sphinxAtStartPar
Asegúrate de haber instalado todos los paquetes y dependencias necesarios antes de continuar.
\end{sphinxadmonition}

\sphinxstepscope


\section{Introducción}
\label{\detokenize{introduccion:introduccion}}\label{\detokenize{introduccion::doc}}
\sphinxAtStartPar
Este proyecto es parte de la asignatura de robots móviles en la Universidad de Alicante y tiene como objetivo explorar conceptos clave de la programación de robots móviles mediante el uso de ROS.

\sphinxAtStartPar
\sphinxstylestrong{Características principales:}
\begin{itemize}
\item {} 
\sphinxAtStartPar
Coordinación mediante máquinas de estados (SMACH).

\item {} 
\sphinxAtStartPar
Procesamiento de imágenes RGBD para detección de objetos.

\item {} 
\sphinxAtStartPar
Navegación autónoma utilizando LIDAR.

\item {} 
\sphinxAtStartPar
Interfaz gráfica para interactuar con el robot.

\end{itemize}

\sphinxstepscope


\section{Configuración del Entorno}
\label{\detokenize{configuracion:configuracion-del-entorno}}\label{\detokenize{configuracion::doc}}\begin{enumerate}
\sphinxsetlistlabels{\arabic}{enumi}{enumii}{}{.}%
\item {} 
\sphinxAtStartPar
\sphinxstylestrong{Crear el Workspace de ROS:}
Si no tienes un workspace configurado, sigue estos pasos:

\begin{sphinxVerbatim}[commandchars=\\\{\}]
mkdir\PYG{+w}{ }\PYGZhy{}p\PYG{+w}{ }rob\PYGZus{}mov\PYGZus{}ws/src
\PYG{n+nb}{cd}\PYG{+w}{ }rob\PYGZus{}mov\PYGZus{}ws/src
git\PYG{+w}{ }clone\PYG{+w}{ }https://github.com/ottocol/navigation\PYGZus{}stage.git
\PYG{n+nb}{cd}\PYG{+w}{ }..
catkin\PYGZus{}make
\PYG{n+nb}{source}\PYG{+w}{ }devel/setup.bash
\end{sphinxVerbatim}

\item {} 
\sphinxAtStartPar
\sphinxstylestrong{Dependencias:}
Asegúrate de instalar las dependencias necesarias para SMACH y otros paquetes ROS:

\begin{sphinxVerbatim}[commandchars=\\\{\}]
sudo\PYG{+w}{ }apt\PYGZhy{}get\PYG{+w}{ }install\PYG{+w}{ }ros\PYGZhy{}noetic\PYGZhy{}smach\PYG{+w}{ }ros\PYGZhy{}noetic\PYGZhy{}smach\PYGZhy{}ros\PYG{+w}{ }ros\PYGZhy{}noetic\PYGZhy{}executive\PYGZhy{}smach\PYG{+w}{ }ros\PYGZhy{}noetic\PYGZhy{}smach\PYGZhy{}viewer
\end{sphinxVerbatim}

\item {} 
\sphinxAtStartPar
\sphinxstylestrong{Lanzar el Simulador:}
Para iniciar el simulador con el stack de navegación:

\begin{sphinxVerbatim}[commandchars=\\\{\}]
roslaunch\PYG{+w}{ }navigation\PYGZus{}stage\PYG{+w}{ }mi\PYGZus{}navigation.launch
\end{sphinxVerbatim}

\end{enumerate}

\sphinxstepscope


\section{Uso del Sistema}
\label{\detokenize{uso:uso-del-sistema}}\label{\detokenize{uso::doc}}
\sphinxAtStartPar
Este sistema implementa una máquina de estados finitos (SMACH) para gestionar diferentes comportamientos del robot. Además, proporciona una interfaz gráfica basada en Tkinter para facilitar la interacción con el robot.


\subsection{Pasos para Lanzar el Sistema}
\label{\detokenize{uso:pasos-para-lanzar-el-sistema}}
\sphinxAtStartPar
Para lanzar el sistema completo, ejecuta el archivo \sphinxtitleref{main.launch} con el siguiente comando:

\begin{sphinxVerbatim}[commandchars=\\\{\}]
roslaunch\PYG{+w}{ }squad\PYGZus{}main\PYG{+w}{ }main.launch
\end{sphinxVerbatim}

\sphinxAtStartPar
Este archivo lanzará todos los nodos necesarios, incluidos los siguientes:
\begin{itemize}
\item {} 
\sphinxAtStartPar
\sphinxstylestrong{Máquina de Estados (TurtleBot State Manager):}
Nodo principal que gestiona los estados del robot y publica el estado actual en el tópico \sphinxtitleref{/current\_state}.

\item {} 
\sphinxAtStartPar
\sphinxstylestrong{Servidores de Acción:}
\sphinxhyphen{} \sphinxtitleref{object\_detection} (detección de objetos).
\sphinxhyphen{} \sphinxtitleref{approach\_object} (aproximación a objetos).
\sphinxhyphen{} \sphinxtitleref{autonomous\_control} (navegación autónoma).

\item {} 
\sphinxAtStartPar
\sphinxstylestrong{Interfaz Gráfica:}
Proporciona una ventana para el control manual del robot y la visualización de imágenes.

\end{itemize}


\subsection{Flujo del Sistema}
\label{\detokenize{uso:flujo-del-sistema}}\begin{enumerate}
\sphinxsetlistlabels{\arabic}{enumi}{enumii}{}{.}%
\item {} 
\sphinxAtStartPar
\sphinxstylestrong{Estado Inicial (Reposo):}
El robot comienza en reposo esperando instrucciones.

\item {} 
\sphinxAtStartPar
\sphinxstylestrong{Exploración:}
Activa el servidor de detección de objetos y supervisa el entorno en busca de objetos.

\item {} 
\sphinxAtStartPar
\sphinxstylestrong{Aproximación:}
Si se detecta un objeto durante la exploración, el robot se aproxima a él utilizando el servidor de acción correspondiente. Este se activa automáticametne desde el estado de exploración si lo hemos activado en la configuración.

\item {} 
\sphinxAtStartPar
\sphinxstylestrong{Navegación:}
El robot se desplaza a una posición predefinida (por ejemplo, «HOME»). Este comando se activa automaticamente al acabar el estado de exploración si lo hemos activado en la configuración.

\end{enumerate}


\subsection{Detalles de la Interfaz Gráfica}
\label{\detokenize{uso:detalles-de-la-interfaz-grafica}}
\sphinxAtStartPar
La interfaz gráfica permite controlar y monitorizar el estado del robot de manera interactiva. Las funcionalidades principales incluyen:
\begin{itemize}
\item {} 
\sphinxAtStartPar
\sphinxstylestrong{Botones de Estado:}
Cambia entre los diferentes estados del robot (reposo, exploración, navegación, aproximación).

\item {} 
\sphinxAtStartPar
\sphinxstylestrong{Visualización de la Cámara:}
Muestra en tiempo real las imágenes captadas por la cámara RGB.

\item {} 
\sphinxAtStartPar
\sphinxstylestrong{Controles Manuales:}
Permite mover el robot manualmente utilizando las teclas de flecha del teclado.

\end{itemize}


\subsection{Configuración de la Máquina de Estados}
\label{\detokenize{uso:configuracion-de-la-maquina-de-estados}}
\sphinxAtStartPar
La máquina de estados está compuesta por los siguientes estados:
\begin{enumerate}
\sphinxsetlistlabels{\arabic}{enumi}{enumii}{}{.}%
\item {} 
\sphinxAtStartPar
\sphinxstylestrong{Reposo:}
\sphinxhyphen{} Publica el estado actual en \sphinxtitleref{/current\_state}.
\sphinxhyphen{} Permite las transiciones a \sphinxtitleref{Exploración} o \sphinxtitleref{Navegación} mediante comandos en el tópico \sphinxtitleref{/command}.

\item {} 
\sphinxAtStartPar
\sphinxstylestrong{Exploración:}
\sphinxhyphen{} Activa la detección de objetos mediante el servidor de acción \sphinxtitleref{object\_detection}.
\sphinxhyphen{} Permite las transiciones a \sphinxtitleref{Reposo}, \sphinxtitleref{Aproximación}, o \sphinxtitleref{Navegación} según los eventos detectados.

\item {} 
\sphinxAtStartPar
\sphinxstylestrong{Aproximación:}
\sphinxhyphen{} Acerca el robot a un objeto detectado utilizando el servidor de acción \sphinxtitleref{approach\_object}.
\sphinxhyphen{} Permite transiciones a \sphinxtitleref{Reposo}, \sphinxtitleref{Exploración}, o \sphinxtitleref{Navegación}.

\item {} 
\sphinxAtStartPar
\sphinxstylestrong{Navegación:}
\sphinxhyphen{} Mueve el robot a una posición predefinida utilizando el servidor de acción \sphinxtitleref{move\_base}.
\sphinxhyphen{} Permite transiciones a \sphinxtitleref{Reposo} o \sphinxtitleref{Exploración}.

\end{enumerate}


\subsection{Comandos y Transiciones}
\label{\detokenize{uso:comandos-y-transiciones}}
\sphinxAtStartPar
Los comandos para cambiar de estado se publican en el tópico \sphinxtitleref{/command}. Algunos ejemplos incluyen:
\begin{itemize}
\item {} 
\sphinxAtStartPar
\sphinxstylestrong{\textasciigrave{}modo exploracion\textasciigrave{}}: Cambia al estado de exploración.

\item {} 
\sphinxAtStartPar
\sphinxstylestrong{\textasciigrave{}ir a HOME\textasciigrave{}}: Cambia al estado de navegación hacia la posición «HOME».

\item {} 
\sphinxAtStartPar
\sphinxstylestrong{\textasciigrave{}acercarse\_objetivo\textasciigrave{}}: Cambia al estado de aproximación hacia un objeto detectado.

\item {} 
\sphinxAtStartPar
\sphinxstylestrong{\textasciigrave{}parar\textasciigrave{}}: Regresa al estado de reposo.

\end{itemize}


\subsection{Parámetros Configurables}
\label{\detokenize{uso:parametros-configurables}}
\sphinxAtStartPar
Los parámetros del sistema se configuran en archivos YAML, que incluyen:
\begin{itemize}
\item {} 
\sphinxAtStartPar
\sphinxstylestrong{\textasciigrave{}configGeneral.yaml\textasciigrave{}}:
Contiene configuraciones globales, como frecuencias de procesamiento, velocidades y distancias mínimas.

\item {} 
\sphinxAtStartPar
\sphinxstylestrong{\textasciigrave{}config.yaml\textasciigrave{}}:
Específico para cada estado, define parámetros como umbrales de detección y destinos predefinidos.

\end{itemize}


\subsection{Tópicos Importantes}
\label{\detokenize{uso:topicos-importantes}}\begin{itemize}
\item {} 
\sphinxAtStartPar
\sphinxstylestrong{\textasciigrave{}/current\_state\textasciigrave{}}:
Publica el estado actual del robot.

\item {} 
\sphinxAtStartPar
\sphinxstylestrong{\textasciigrave{}/command\textasciigrave{}}:
Recibe comandos para cambiar el estado de la máquina de estados.

\item {} 
\sphinxAtStartPar
\sphinxstylestrong{\textasciigrave{}/detected\_objects\textasciigrave{}}:
Publica la información de los objetos detectados.

\item {} 
\sphinxAtStartPar
\sphinxstylestrong{\textasciigrave{}/cmd\_vel\textasciigrave{}}:
Envia comandos de movimiento al robot.

\item {} 
\sphinxAtStartPar
\sphinxstylestrong{\textasciigrave{}/camera/rgb/image\_raw\textasciigrave{}}:
Proporciona imágenes RGB de la cámara.

\end{itemize}

\sphinxAtStartPar
—

\sphinxAtStartPar
Con esta guía, deberías tener todo lo necesario para ejecutar, configurar y controlar el sistema. Si necesitas más información, consulta la sección de configuración o los detalles específicos de los nodos en la documentación. 😊

\sphinxstepscope


\section{Máquina de Estados (squad\_state\_manager)}
\label{\detokenize{squad_state_manager:maquina-de-estados-squad-state-manager}}\label{\detokenize{squad_state_manager::doc}}
\sphinxAtStartPar
Descripción general del módulo \sphinxtitleref{squad\_state\_manager}.

\sphinxAtStartPar
\sphinxstylestrong{Estados principales:}
\begin{itemize}
\item {} 
\sphinxAtStartPar
\sphinxstylestrong{Reposo:} Estado inicial de espera.

\item {} 
\sphinxAtStartPar
\sphinxstylestrong{Exploración:} Activación del servidor de detección de objetos.

\item {} 
\sphinxAtStartPar
\sphinxstylestrong{Aproximación:} Movimiento hacia un objeto detectado.

\item {} 
\sphinxAtStartPar
\sphinxstylestrong{Navegación:} Desplazamiento a un punto específico.

\end{itemize}

\sphinxAtStartPar
\sphinxstylestrong{Transiciones:}
\sphinxhyphen{} De \sphinxtitleref{Reposo} a \sphinxtitleref{Exploración} tras recibir el comando correspondiente.
\sphinxhyphen{} De \sphinxtitleref{Exploración} a \sphinxtitleref{Aproximación} tras detectar un objeto.
\sphinxhyphen{} De \sphinxtitleref{Aproximación} a \sphinxtitleref{Reposo} si se alcanza el objetivo o tras cierto tiempo.

\sphinxincludegraphics[]{graphviz-6b0cff7ad1f1912f03aecf761b994e27935307b8.pdf}
\phantomsection\label{\detokenize{squad_state_manager:module-squad_state_manager}}\index{module@\spxentry{module}!squad\_state\_manager@\spxentry{squad\_state\_manager}}\index{squad\_state\_manager@\spxentry{squad\_state\_manager}!module@\spxentry{module}}
\sphinxAtStartPar
TurtleBot State Manager

\sphinxAtStartPar
Este script gestiona el estado del TurtleBot utilizando SMACH, una biblioteca de Python
para construir máquinas de estados finitas. Define y maneja diferentes estados como
reposo, exploración, navegación y aproximación a objetos detectados. Además, integra
una interfaz gráfica de usuario (GUI) construida con Tkinter para permitir la interacción
manual y visualización del estado actual del robot y la imagen de la cámara.
\begin{description}
\sphinxlineitem{Dependencias:}\begin{itemize}
\item {} 
\sphinxAtStartPar
rospy

\item {} 
\sphinxAtStartPar
smach

\item {} 
\sphinxAtStartPar
smach\_ros

\item {} 
\sphinxAtStartPar
actionlib

\item {} 
\sphinxAtStartPar
std\_msgs.msg

\item {} 
\sphinxAtStartPar
squad\_exploracion.msg

\item {} 
\sphinxAtStartPar
sensor\_msgs.msg

\item {} 
\sphinxAtStartPar
cv\_bridge

\item {} 
\sphinxAtStartPar
PIL (Pillow)

\item {} 
\sphinxAtStartPar
cv2 (OpenCV)

\item {} 
\sphinxAtStartPar
move\_base\_msgs.msg

\item {} 
\sphinxAtStartPar
geometry\_msgs.msg

\item {} 
\sphinxAtStartPar
threading

\item {} 
\sphinxAtStartPar
tkinter

\end{itemize}

\end{description}
\index{BaseState (clase en squad\_state\_manager)@\spxentry{BaseState}\spxextra{clase en squad\_state\_manager}}

\begin{fulllineitems}
\phantomsection\label{\detokenize{squad_state_manager:squad_state_manager.BaseState}}
\pysigstartsignatures
\pysiglinewithargsret{\sphinxbfcode{\sphinxupquote{class\DUrole{w,w}{  }}}\sphinxcode{\sphinxupquote{squad\_state\_manager.}}\sphinxbfcode{\sphinxupquote{BaseState}}}{\sphinxparam{\DUrole{o,o}{**}\DUrole{n,n}{kwargs}}}{}
\pysigstopsignatures
\sphinxAtStartPar
Bases: \sphinxcode{\sphinxupquote{State}}

\sphinxAtStartPar
Clase base para todos los estados que publica el estado actual.

\sphinxAtStartPar
Esta clase sirve como clase base para los diferentes estados de la máquina de estados finita (SMACH).
Publica el estado actual y suscribe a comandos para gestionar transiciones entre estados.
\begin{description}
\sphinxlineitem{Hereda de:}
\sphinxAtStartPar
smach.State: Clase base proporcionada por SMACH para definir estados.

\sphinxlineitem{Atributos:}
\sphinxAtStartPar
state\_pub (Publisher): Publicador al tópico “/current\_state” para publicar el estado actual.
command\_sub (Subscriber): Suscriptor al tópico “/command” para recibir comandos de transición.
comando (str): Comando actual recibido desde el tópico “/command”.
satate\_name (str): Nombre del estado actual.

\end{description}
\index{execute() (método de squad\_state\_manager.BaseState)@\spxentry{execute()}\spxextra{método de squad\_state\_manager.BaseState}}

\begin{fulllineitems}
\phantomsection\label{\detokenize{squad_state_manager:squad_state_manager.BaseState.execute}}
\pysigstartsignatures
\pysiglinewithargsret{\sphinxbfcode{\sphinxupquote{execute}}}{\sphinxparam{\DUrole{n,n}{userdata}}}{}
\pysigstopsignatures
\sphinxAtStartPar
Publica el nombre del estado actual.

\sphinxAtStartPar
Este método debe ser implementado en las subclases para definir el comportamiento específico del estado.
\begin{quote}\begin{description}
\sphinxlineitem{Parámetros}
\sphinxAtStartPar
\sphinxstyleliteralstrong{\sphinxupquote{userdata}} \textendash{} Datos de usuario proporcionados por SMACH.

\sphinxlineitem{Devuelve}
\sphinxAtStartPar
Resultado del estado que determina la transición siguiente.

\sphinxlineitem{Tipo del valor devuelto}
\sphinxAtStartPar
\sphinxhref{https://docs.python.org/3/library/stdtypes.html\#str}{str}

\end{description}\end{quote}

\end{fulllineitems}

\index{state\_callback() (método de squad\_state\_manager.BaseState)@\spxentry{state\_callback()}\spxextra{método de squad\_state\_manager.BaseState}}

\begin{fulllineitems}
\phantomsection\label{\detokenize{squad_state_manager:squad_state_manager.BaseState.state_callback}}
\pysigstartsignatures
\pysiglinewithargsret{\sphinxbfcode{\sphinxupquote{state\_callback}}}{\sphinxparam{\DUrole{n,n}{msg}}}{}
\pysigstopsignatures
\sphinxAtStartPar
Callback para recibir comandos desde el tópico “/command”.

\sphinxAtStartPar
Actualiza el atributo “comando” con el comando recibido en minúsculas.
\begin{quote}\begin{description}
\sphinxlineitem{Parámetros}
\sphinxAtStartPar
\sphinxstyleliteralstrong{\sphinxupquote{msg}} (\sphinxstyleliteralemphasis{\sphinxupquote{String}}) \textendash{} Mensaje con el comando recibido.

\end{description}\end{quote}

\end{fulllineitems}


\end{fulllineitems}

\index{EstadoApproach (clase en squad\_state\_manager)@\spxentry{EstadoApproach}\spxextra{clase en squad\_state\_manager}}

\begin{fulllineitems}
\phantomsection\label{\detokenize{squad_state_manager:squad_state_manager.EstadoApproach}}
\pysigstartsignatures
\pysigline{\sphinxbfcode{\sphinxupquote{class\DUrole{w,w}{  }}}\sphinxcode{\sphinxupquote{squad\_state\_manager.}}\sphinxbfcode{\sphinxupquote{EstadoApproach}}}
\pysigstopsignatures
\sphinxAtStartPar
Bases: {\hyperref[\detokenize{squad_state_manager:squad_state_manager.BaseState}]{\sphinxcrossref{\sphinxcode{\sphinxupquote{BaseState}}}}}

\sphinxAtStartPar
Estado de Aproximación del robot que maneja el acercamiento a objetos detectados.

\sphinxAtStartPar
Este estado gestiona la lógica para que el robot se acerque a un objeto detectado utilizando
un servidor de acción específico para el control de aproximación.
\begin{description}
\sphinxlineitem{Hereda de:}
\sphinxAtStartPar
BaseState: Clase base para estados en la máquina de estados finita (SMACH).

\sphinxlineitem{Atributos:}
\sphinxAtStartPar
client\_approach (SimpleActionClient): Cliente de acción para el control de aproximación.
detectar\_multiples\_objetos (bool): Indicador para detectar múltiples objetos.
acercarse\_objetos (bool): Indicador para acercarse a objetos detectados.
volver\_casa (bool): Indicador para volver a la posición «HOME».
last\_object\_detected (DetectedObject): Último objeto detectado.
comando (str): Comando actual recibido.

\end{description}
\index{control\_mode\_callback() (método de squad\_state\_manager.EstadoApproach)@\spxentry{control\_mode\_callback()}\spxextra{método de squad\_state\_manager.EstadoApproach}}

\begin{fulllineitems}
\phantomsection\label{\detokenize{squad_state_manager:squad_state_manager.EstadoApproach.control_mode_callback}}
\pysigstartsignatures
\pysiglinewithargsret{\sphinxbfcode{\sphinxupquote{control\_mode\_callback}}}{\sphinxparam{\DUrole{n,n}{msg}}}{}
\pysigstopsignatures
\sphinxAtStartPar
Callback para el suscriptor del modo de control.

\sphinxAtStartPar
Cambia el modo de control del robot entre “manual” y “autonomous”.
\begin{quote}\begin{description}
\sphinxlineitem{Parámetros}
\sphinxAtStartPar
\sphinxstyleliteralstrong{\sphinxupquote{msg}} (\sphinxstyleliteralemphasis{\sphinxupquote{String}}) \textendash{} Mensaje con el modo de control.

\end{description}\end{quote}

\end{fulllineitems}

\index{execute() (método de squad\_state\_manager.EstadoApproach)@\spxentry{execute()}\spxextra{método de squad\_state\_manager.EstadoApproach}}

\begin{fulllineitems}
\phantomsection\label{\detokenize{squad_state_manager:squad_state_manager.EstadoApproach.execute}}
\pysigstartsignatures
\pysiglinewithargsret{\sphinxbfcode{\sphinxupquote{execute}}}{\sphinxparam{\DUrole{n,n}{userdata}}}{}
\pysigstopsignatures
\sphinxAtStartPar
Ejecuta el comportamiento principal del estado de Aproximación.

\sphinxAtStartPar
Este método gestiona la lógica para acercarse a un objeto detectado y las transiciones
hacia otros estados basándose en el resultado de la aproximación.
\begin{quote}\begin{description}
\sphinxlineitem{Parámetros}
\sphinxAtStartPar
\sphinxstyleliteralstrong{\sphinxupquote{userdata}} \textendash{} Datos de usuario proporcionados por SMACH.

\sphinxlineitem{Devuelve}
\sphinxAtStartPar
El resultado del estado, que determina la transición siguiente.

\sphinxlineitem{Tipo del valor devuelto}
\sphinxAtStartPar
\sphinxhref{https://docs.python.org/3/library/stdtypes.html\#str}{str}

\end{description}\end{quote}

\end{fulllineitems}

\index{feedback\_cb() (método de squad\_state\_manager.EstadoApproach)@\spxentry{feedback\_cb()}\spxextra{método de squad\_state\_manager.EstadoApproach}}

\begin{fulllineitems}
\phantomsection\label{\detokenize{squad_state_manager:squad_state_manager.EstadoApproach.feedback_cb}}
\pysigstartsignatures
\pysiglinewithargsret{\sphinxbfcode{\sphinxupquote{feedback\_cb}}}{\sphinxparam{\DUrole{n,n}{feedback}}}{}
\pysigstopsignatures
\sphinxAtStartPar
Callback para recibir feedback del servidor de aproximación.

\sphinxAtStartPar
Actualmente, este callback no realiza ninguna acción adicional.
\begin{quote}\begin{description}
\sphinxlineitem{Parámetros}
\sphinxAtStartPar
\sphinxstyleliteralstrong{\sphinxupquote{feedback}} \textendash{} Feedback recibido del servidor de acción de aproximación.

\end{description}\end{quote}

\end{fulllineitems}


\end{fulllineitems}

\index{EstadoExploracion (clase en squad\_state\_manager)@\spxentry{EstadoExploracion}\spxextra{clase en squad\_state\_manager}}

\begin{fulllineitems}
\phantomsection\label{\detokenize{squad_state_manager:squad_state_manager.EstadoExploracion}}
\pysigstartsignatures
\pysigline{\sphinxbfcode{\sphinxupquote{class\DUrole{w,w}{  }}}\sphinxcode{\sphinxupquote{squad\_state\_manager.}}\sphinxbfcode{\sphinxupquote{EstadoExploracion}}}
\pysigstopsignatures
\sphinxAtStartPar
Bases: {\hyperref[\detokenize{squad_state_manager:squad_state_manager.BaseState}]{\sphinxcrossref{\sphinxcode{\sphinxupquote{BaseState}}}}}

\sphinxAtStartPar
Estado de Exploración del robot que gestiona la detección de objetos y el control autónomo.

\sphinxAtStartPar
Este estado coordina los comportamientos del robot durante la fase de exploración, incluyendo la
detección de objetos, la aproximación a objetos detectados y la navegación autónoma.
\begin{description}
\sphinxlineitem{Hereda de:}
\sphinxAtStartPar
BaseState: Clase base para estados en la máquina de estados finita (SMACH).

\sphinxlineitem{Atributos:}
\sphinxAtStartPar
client\_object\_detection (SimpleActionClient): Cliente de acción para la detección de objetos.
client\_autonomous\_control (SimpleActionClient): Cliente de acción para el control autónomo.
object\_sub (Subscriber): Suscriptor al tópico de objetos detectados.
state\_sub (Subscriber): Suscriptor al tópico del estado actual.
control\_mode\_sub (Subscriber): Suscriptor al tópico del modo de control.
feedback\_timeout (Duration): Tiempo máximo de espera para recibir feedback de detección.
last\_feedback\_time (Time): Última vez que se recibió feedback.
detectar\_multiples\_objetos (bool): Indicador para detectar múltiples objetos.
acercarse\_objetos (bool): Indicador para aproximarse a objetos detectados.
volver\_casa (bool): Indicador para volver a la posición «HOME».
comando (str): Comando actual recibido.
control\_mode (str): Modo de control actual (“manual” o “autonomous”).
autonomous\_control\_active (bool): Estado de la acción de control autónomo.
objeto\_detectado (bool): Indicador de detección de objeto.
last\_object\_detected (DetectedObject): Último objeto detectado.

\end{description}
\index{control\_mode\_callback() (método de squad\_state\_manager.EstadoExploracion)@\spxentry{control\_mode\_callback()}\spxextra{método de squad\_state\_manager.EstadoExploracion}}

\begin{fulllineitems}
\phantomsection\label{\detokenize{squad_state_manager:squad_state_manager.EstadoExploracion.control_mode_callback}}
\pysigstartsignatures
\pysiglinewithargsret{\sphinxbfcode{\sphinxupquote{control\_mode\_callback}}}{\sphinxparam{\DUrole{n,n}{msg}}}{}
\pysigstopsignatures
\sphinxAtStartPar
Callback para el suscriptor del modo de control.

\sphinxAtStartPar
Cambia el modo de control del robot entre “manual” y “autonomous”.
\begin{quote}\begin{description}
\sphinxlineitem{Parámetros}
\sphinxAtStartPar
\sphinxstyleliteralstrong{\sphinxupquote{msg}} (\sphinxstyleliteralemphasis{\sphinxupquote{String}}) \textendash{} Mensaje con el modo de control.

\end{description}\end{quote}

\end{fulllineitems}

\index{execute() (método de squad\_state\_manager.EstadoExploracion)@\spxentry{execute()}\spxextra{método de squad\_state\_manager.EstadoExploracion}}

\begin{fulllineitems}
\phantomsection\label{\detokenize{squad_state_manager:squad_state_manager.EstadoExploracion.execute}}
\pysigstartsignatures
\pysiglinewithargsret{\sphinxbfcode{\sphinxupquote{execute}}}{\sphinxparam{\DUrole{n,n}{userdata}}}{}
\pysigstopsignatures
\sphinxAtStartPar
Ejecuta el comportamiento principal del estado de Exploración.

\sphinxAtStartPar
Este método gestiona la lógica de detección de objetos, control autónomo,
y las transiciones entre diferentes estados basándose en eventos y comandos.
\begin{quote}\begin{description}
\sphinxlineitem{Parámetros}
\sphinxAtStartPar
\sphinxstyleliteralstrong{\sphinxupquote{userdata}} \textendash{} Datos de usuario proporcionados por SMACH.

\sphinxlineitem{Devuelve}
\sphinxAtStartPar
El resultado del estado, que determina la transición siguiente.

\sphinxlineitem{Tipo del valor devuelto}
\sphinxAtStartPar
\sphinxhref{https://docs.python.org/3/library/stdtypes.html\#str}{str}

\end{description}\end{quote}

\end{fulllineitems}

\index{feedback\_cb() (método de squad\_state\_manager.EstadoExploracion)@\spxentry{feedback\_cb()}\spxextra{método de squad\_state\_manager.EstadoExploracion}}

\begin{fulllineitems}
\phantomsection\label{\detokenize{squad_state_manager:squad_state_manager.EstadoExploracion.feedback_cb}}
\pysigstartsignatures
\pysiglinewithargsret{\sphinxbfcode{\sphinxupquote{feedback\_cb}}}{\sphinxparam{\DUrole{n,n}{feedback}}}{}
\pysigstopsignatures
\sphinxAtStartPar
Callback para recibir feedback del servidor de detección de objetos.

\sphinxAtStartPar
Actualiza el tiempo del último feedback recibido.
\begin{quote}\begin{description}
\sphinxlineitem{Parámetros}
\sphinxAtStartPar
\sphinxstyleliteralstrong{\sphinxupquote{feedback}} \textendash{} Feedback recibido del servidor de acción de detección de objetos.

\end{description}\end{quote}

\end{fulllineitems}

\index{object\_callback() (método de squad\_state\_manager.EstadoExploracion)@\spxentry{object\_callback()}\spxextra{método de squad\_state\_manager.EstadoExploracion}}

\begin{fulllineitems}
\phantomsection\label{\detokenize{squad_state_manager:squad_state_manager.EstadoExploracion.object_callback}}
\pysigstartsignatures
\pysiglinewithargsret{\sphinxbfcode{\sphinxupquote{object\_callback}}}{\sphinxparam{\DUrole{n,n}{msg}}}{}
\pysigstopsignatures
\sphinxAtStartPar
Callback para el suscriptor de objetos detectados.

\sphinxAtStartPar
Actualiza el estado interno cuando se detecta un objeto.
\begin{quote}\begin{description}
\sphinxlineitem{Parámetros}
\sphinxAtStartPar
\sphinxstyleliteralstrong{\sphinxupquote{msg}} (\sphinxstyleliteralemphasis{\sphinxupquote{DetectedObject}}) \textendash{} Mensaje con la información del objeto detectado.

\end{description}\end{quote}

\end{fulllineitems}


\end{fulllineitems}

\index{EstadoNavegacion (clase en squad\_state\_manager)@\spxentry{EstadoNavegacion}\spxextra{clase en squad\_state\_manager}}

\begin{fulllineitems}
\phantomsection\label{\detokenize{squad_state_manager:squad_state_manager.EstadoNavegacion}}
\pysigstartsignatures
\pysigline{\sphinxbfcode{\sphinxupquote{class\DUrole{w,w}{  }}}\sphinxcode{\sphinxupquote{squad\_state\_manager.}}\sphinxbfcode{\sphinxupquote{EstadoNavegacion}}}
\pysigstopsignatures
\sphinxAtStartPar
Bases: {\hyperref[\detokenize{squad_state_manager:squad_state_manager.BaseState}]{\sphinxcrossref{\sphinxcode{\sphinxupquote{BaseState}}}}}

\sphinxAtStartPar
Estado de Navegación del robot que maneja el desplazamiento a una posición predefinida.

\sphinxAtStartPar
Este estado utiliza el servidor de acción \sphinxtitleref{move\_base} para mover el robot a una ubicación específica
definida en un diccionario de estaciones.
\begin{description}
\sphinxlineitem{Hereda de:}
\sphinxAtStartPar
BaseState: Clase base para estados en la máquina de estados finita (SMACH).

\sphinxlineitem{Atributos:}
\sphinxAtStartPar
client (SimpleActionClient): Cliente de acción para \sphinxtitleref{move\_base}.
satate\_name (str): Nombre del estado actual.
comando (str): Comando actual recibido.

\end{description}
\index{control\_mode\_callback() (método de squad\_state\_manager.EstadoNavegacion)@\spxentry{control\_mode\_callback()}\spxextra{método de squad\_state\_manager.EstadoNavegacion}}

\begin{fulllineitems}
\phantomsection\label{\detokenize{squad_state_manager:squad_state_manager.EstadoNavegacion.control_mode_callback}}
\pysigstartsignatures
\pysiglinewithargsret{\sphinxbfcode{\sphinxupquote{control\_mode\_callback}}}{\sphinxparam{\DUrole{n,n}{msg}}}{}
\pysigstopsignatures
\sphinxAtStartPar
Callback para el suscriptor del modo de control.

\sphinxAtStartPar
Cambia el modo de control del robot entre “manual” y “autonomous”.
\begin{quote}\begin{description}
\sphinxlineitem{Parámetros}
\sphinxAtStartPar
\sphinxstyleliteralstrong{\sphinxupquote{msg}} (\sphinxstyleliteralemphasis{\sphinxupquote{String}}) \textendash{} Mensaje con el modo de control.

\end{description}\end{quote}

\end{fulllineitems}

\index{execute() (método de squad\_state\_manager.EstadoNavegacion)@\spxentry{execute()}\spxextra{método de squad\_state\_manager.EstadoNavegacion}}

\begin{fulllineitems}
\phantomsection\label{\detokenize{squad_state_manager:squad_state_manager.EstadoNavegacion.execute}}
\pysigstartsignatures
\pysiglinewithargsret{\sphinxbfcode{\sphinxupquote{execute}}}{\sphinxparam{\DUrole{n,n}{userdata}}}{}
\pysigstopsignatures
\sphinxAtStartPar
Ejecuta el comportamiento principal del estado de Navegación.

\sphinxAtStartPar
Este método envía un objetivo al servidor de acción \sphinxtitleref{move\_base} para mover el robot
a las coordenadas especificadas en el destino y gestiona las transiciones basadas
en el resultado de la navegación.
\begin{quote}\begin{description}
\sphinxlineitem{Parámetros}
\sphinxAtStartPar
\sphinxstyleliteralstrong{\sphinxupquote{userdata}} \textendash{} Datos de usuario proporcionados por SMACH.

\sphinxlineitem{Devuelve}
\sphinxAtStartPar
El resultado del estado, que determina la transición siguiente.

\sphinxlineitem{Tipo del valor devuelto}
\sphinxAtStartPar
\sphinxhref{https://docs.python.org/3/library/stdtypes.html\#str}{str}

\end{description}\end{quote}

\end{fulllineitems}

\index{moveTo() (método de squad\_state\_manager.EstadoNavegacion)@\spxentry{moveTo()}\spxextra{método de squad\_state\_manager.EstadoNavegacion}}

\begin{fulllineitems}
\phantomsection\label{\detokenize{squad_state_manager:squad_state_manager.EstadoNavegacion.moveTo}}
\pysigstartsignatures
\pysiglinewithargsret{\sphinxbfcode{\sphinxupquote{moveTo}}}{\sphinxparam{\DUrole{n,n}{x}}\sphinxparamcomma \sphinxparam{\DUrole{n,n}{y}}}{}
\pysigstopsignatures
\sphinxAtStartPar
Envía un objetivo al servidor de acción \sphinxtitleref{move\_base} para mover el robot a las coordenadas especificadas.
\begin{quote}\begin{description}
\sphinxlineitem{Parámetros}\begin{itemize}
\item {} 
\sphinxAtStartPar
\sphinxstyleliteralstrong{\sphinxupquote{x}} (\sphinxhref{https://docs.python.org/3/library/functions.html\#float}{\sphinxstyleliteralemphasis{\sphinxupquote{float}}}) \textendash{} Coordenada X del destino.

\item {} 
\sphinxAtStartPar
\sphinxstyleliteralstrong{\sphinxupquote{y}} (\sphinxhref{https://docs.python.org/3/library/functions.html\#float}{\sphinxstyleliteralemphasis{\sphinxupquote{float}}}) \textendash{} Coordenada Y del destino.

\end{itemize}

\end{description}\end{quote}

\end{fulllineitems}


\end{fulllineitems}

\index{EstadoReposo (clase en squad\_state\_manager)@\spxentry{EstadoReposo}\spxextra{clase en squad\_state\_manager}}

\begin{fulllineitems}
\phantomsection\label{\detokenize{squad_state_manager:squad_state_manager.EstadoReposo}}
\pysigstartsignatures
\pysigline{\sphinxbfcode{\sphinxupquote{class\DUrole{w,w}{  }}}\sphinxcode{\sphinxupquote{squad\_state\_manager.}}\sphinxbfcode{\sphinxupquote{EstadoReposo}}}
\pysigstopsignatures
\sphinxAtStartPar
Bases: {\hyperref[\detokenize{squad_state_manager:squad_state_manager.BaseState}]{\sphinxcrossref{\sphinxcode{\sphinxupquote{BaseState}}}}}
\index{execute() (método de squad\_state\_manager.EstadoReposo)@\spxentry{execute()}\spxextra{método de squad\_state\_manager.EstadoReposo}}

\begin{fulllineitems}
\phantomsection\label{\detokenize{squad_state_manager:squad_state_manager.EstadoReposo.execute}}
\pysigstartsignatures
\pysiglinewithargsret{\sphinxbfcode{\sphinxupquote{execute}}}{\sphinxparam{\DUrole{n,n}{userdata}}}{}
\pysigstopsignatures
\sphinxAtStartPar
Publica el nombre del estado actual.

\sphinxAtStartPar
Este método debe ser implementado en las subclases para definir el comportamiento específico del estado.
\begin{quote}\begin{description}
\sphinxlineitem{Parámetros}
\sphinxAtStartPar
\sphinxstyleliteralstrong{\sphinxupquote{userdata}} \textendash{} Datos de usuario proporcionados por SMACH.

\sphinxlineitem{Devuelve}
\sphinxAtStartPar
Resultado del estado que determina la transición siguiente.

\sphinxlineitem{Tipo del valor devuelto}
\sphinxAtStartPar
\sphinxhref{https://docs.python.org/3/library/stdtypes.html\#str}{str}

\end{description}\end{quote}

\end{fulllineitems}


\end{fulllineitems}

\index{InterfazManager (clase en squad\_state\_manager)@\spxentry{InterfazManager}\spxextra{clase en squad\_state\_manager}}

\begin{fulllineitems}
\phantomsection\label{\detokenize{squad_state_manager:squad_state_manager.InterfazManager}}
\pysigstartsignatures
\pysiglinewithargsret{\sphinxbfcode{\sphinxupquote{class\DUrole{w,w}{  }}}\sphinxcode{\sphinxupquote{squad\_state\_manager.}}\sphinxbfcode{\sphinxupquote{InterfazManager}}}{\sphinxparam{\DUrole{n,n}{state\_machine}}}{}
\pysigstopsignatures
\sphinxAtStartPar
Bases: \sphinxhref{https://docs.python.org/3/library/functions.html\#object}{\sphinxcode{\sphinxupquote{object}}}

\sphinxAtStartPar
Gestor de la Interfaz Gráfica de Usuario (GUI) para la Máquina de Estados del TurtleBot.

\sphinxAtStartPar
Esta clase maneja la creación y actualización de la interfaz gráfica, la interacción con
los tópicos de ROS para enviar comandos y recibir actualizaciones del estado del robot.
También gestiona la visualización de imágenes de la cámara en tiempo real.
\begin{description}
\sphinxlineitem{Atributos:}
\sphinxAtStartPar
state\_machine (StateMachine): Máquina de estados finita (SMACH) para gestionar los estados del robot.
state\_names (list): Lista de nombres de los estados definidos en la máquina de estados.
current\_state (str): Nombre del estado actual del robot.
command\_state\_mapping (dict): Mapeo de comandos a nombres de estados.
cmd\_vel\_pub (Publisher): Publicador al tópico “/cmd\_vel” para enviar comandos de movimiento.
state\_pub (Publisher): Publicador al tópico “/command” para enviar comandos de estado.
control\_mode\_pub (Publisher): Publicador al tópico “/control\_mode” para cambiar el modo de control.
control\_mode\_sub (Subscriber): Suscriptor al tópico “/control\_mode” para recibir actualizaciones del modo de control.
bridge (CvBridge): Objeto para convertir mensajes de ROS Image a OpenCV.
camera\_sub (Subscriber): Suscriptor al tópico de la cámara.
normal\_camera\_topic (str): Tópico de la cámara normal.
action\_camera\_topic (str): Tópico de la cámara procesada para acciones.
current\_camera\_topic (str): Tópico de la cámara actualmente suscrito.
camera\_image (ImageTk.PhotoImage): Imagen actual de la cámara para visualizar en la GUI.
control\_mode (str): Modo de control actual (“manual” o “autonomous”).
window (Tk): Ventana principal de la GUI.
state\_buttons (dict): Diccionario de botones de estado en la GUI.
exploration\_controls\_frame (Frame): Marco para los controles de exploración en la GUI.

\end{description}
\index{camera\_callback() (método de squad\_state\_manager.InterfazManager)@\spxentry{camera\_callback()}\spxextra{método de squad\_state\_manager.InterfazManager}}

\begin{fulllineitems}
\phantomsection\label{\detokenize{squad_state_manager:squad_state_manager.InterfazManager.camera_callback}}
\pysigstartsignatures
\pysiglinewithargsret{\sphinxbfcode{\sphinxupquote{camera\_callback}}}{\sphinxparam{\DUrole{n,n}{msg}}}{}
\pysigstopsignatures
\sphinxAtStartPar
Callback para el suscriptor de la cámara.

\sphinxAtStartPar
Convierte la imagen recibida de ROS a un formato compatible con Tkinter y la
muestra en la GUI.
\begin{quote}\begin{description}
\sphinxlineitem{Parámetros}
\sphinxAtStartPar
\sphinxstyleliteralstrong{\sphinxupquote{msg}} (\sphinxstyleliteralemphasis{\sphinxupquote{ROSImage}}) \textendash{} Mensaje de imagen recibido del tópico de la cámara.

\end{description}\end{quote}

\end{fulllineitems}

\index{change\_exploration\_mode() (método de squad\_state\_manager.InterfazManager)@\spxentry{change\_exploration\_mode()}\spxextra{método de squad\_state\_manager.InterfazManager}}

\begin{fulllineitems}
\phantomsection\label{\detokenize{squad_state_manager:squad_state_manager.InterfazManager.change_exploration_mode}}
\pysigstartsignatures
\pysiglinewithargsret{\sphinxbfcode{\sphinxupquote{change\_exploration\_mode}}}{}{}
\pysigstopsignatures
\sphinxAtStartPar
Cambia el modo de exploración entre “manual” y “autonomous”.

\sphinxAtStartPar
Actualiza la GUI y publica el nuevo modo de control al tópico “/control\_mode”.

\end{fulllineitems}

\index{control\_mode\_callback() (método de squad\_state\_manager.InterfazManager)@\spxentry{control\_mode\_callback()}\spxextra{método de squad\_state\_manager.InterfazManager}}

\begin{fulllineitems}
\phantomsection\label{\detokenize{squad_state_manager:squad_state_manager.InterfazManager.control_mode_callback}}
\pysigstartsignatures
\pysiglinewithargsret{\sphinxbfcode{\sphinxupquote{control\_mode\_callback}}}{\sphinxparam{\DUrole{n,n}{msg}}}{}
\pysigstopsignatures
\sphinxAtStartPar
Callback para el suscriptor del modo de control.

\sphinxAtStartPar
Actualiza el modo de control actual y realiza acciones adicionales si es necesario.
\begin{quote}\begin{description}
\sphinxlineitem{Parámetros}
\sphinxAtStartPar
\sphinxstyleliteralstrong{\sphinxupquote{msg}} (\sphinxstyleliteralemphasis{\sphinxupquote{String}}) \textendash{} Mensaje con el modo de control (“manual” o “autonomous”).

\end{description}\end{quote}

\end{fulllineitems}

\index{create\_exploration\_controls() (método de squad\_state\_manager.InterfazManager)@\spxentry{create\_exploration\_controls()}\spxextra{método de squad\_state\_manager.InterfazManager}}

\begin{fulllineitems}
\phantomsection\label{\detokenize{squad_state_manager:squad_state_manager.InterfazManager.create_exploration_controls}}
\pysigstartsignatures
\pysiglinewithargsret{\sphinxbfcode{\sphinxupquote{create\_exploration\_controls}}}{}{}
\pysigstopsignatures
\sphinxAtStartPar
Crea los controles de exploración en la interfaz gráfica.

\sphinxAtStartPar
Incluye un botón para cambiar el modo de exploración y las instrucciones para el
control manual mediante el teclado.

\end{fulllineitems}

\index{current\_state\_callback() (método de squad\_state\_manager.InterfazManager)@\spxentry{current\_state\_callback()}\spxextra{método de squad\_state\_manager.InterfazManager}}

\begin{fulllineitems}
\phantomsection\label{\detokenize{squad_state_manager:squad_state_manager.InterfazManager.current_state_callback}}
\pysigstartsignatures
\pysiglinewithargsret{\sphinxbfcode{\sphinxupquote{current\_state\_callback}}}{\sphinxparam{\DUrole{n,n}{msg}}}{}
\pysigstopsignatures
\sphinxAtStartPar
Callback para el suscriptor del estado actual.

\sphinxAtStartPar
Actualiza el estado actual y resalta el botón correspondiente en la GUI.
\begin{quote}\begin{description}
\sphinxlineitem{Parámetros}
\sphinxAtStartPar
\sphinxstyleliteralstrong{\sphinxupquote{msg}} (\sphinxstyleliteralemphasis{\sphinxupquote{String}}) \textendash{} Mensaje con el nombre del estado actual.

\end{description}\end{quote}

\end{fulllineitems}

\index{get\_command\_for\_state() (método de squad\_state\_manager.InterfazManager)@\spxentry{get\_command\_for\_state()}\spxextra{método de squad\_state\_manager.InterfazManager}}

\begin{fulllineitems}
\phantomsection\label{\detokenize{squad_state_manager:squad_state_manager.InterfazManager.get_command_for_state}}
\pysigstartsignatures
\pysiglinewithargsret{\sphinxbfcode{\sphinxupquote{get\_command\_for\_state}}}{\sphinxparam{\DUrole{n,n}{state\_name}}}{}
\pysigstopsignatures
\sphinxAtStartPar
Obtiene el comando correspondiente para un nombre de estado dado.
\begin{quote}\begin{description}
\sphinxlineitem{Parámetros}
\sphinxAtStartPar
\sphinxstyleliteralstrong{\sphinxupquote{state\_name}} (\sphinxhref{https://docs.python.org/3/library/stdtypes.html\#str}{\sphinxstyleliteralemphasis{\sphinxupquote{str}}}) \textendash{} Nombre del estado.

\sphinxlineitem{Devuelve}
\sphinxAtStartPar
Comando asociado al estado o None si no hay mapeo.

\sphinxlineitem{Tipo del valor devuelto}
\sphinxAtStartPar
\sphinxhref{https://docs.python.org/3/library/stdtypes.html\#str}{str} or None

\end{description}\end{quote}

\end{fulllineitems}

\index{on\_arrow\_down() (método de squad\_state\_manager.InterfazManager)@\spxentry{on\_arrow\_down()}\spxextra{método de squad\_state\_manager.InterfazManager}}

\begin{fulllineitems}
\phantomsection\label{\detokenize{squad_state_manager:squad_state_manager.InterfazManager.on_arrow_down}}
\pysigstartsignatures
\pysiglinewithargsret{\sphinxbfcode{\sphinxupquote{on\_arrow\_down}}}{\sphinxparam{\DUrole{n,n}{event}}}{}
\pysigstopsignatures
\sphinxAtStartPar
Maneja la pulsación de la tecla de flecha hacia abajo para mover el robot hacia atrás.
\begin{quote}\begin{description}
\sphinxlineitem{Parámetros}
\sphinxAtStartPar
\sphinxstyleliteralstrong{\sphinxupquote{event}} \textendash{} Evento de la tecla pulsada.

\end{description}\end{quote}

\end{fulllineitems}

\index{on\_arrow\_left() (método de squad\_state\_manager.InterfazManager)@\spxentry{on\_arrow\_left()}\spxextra{método de squad\_state\_manager.InterfazManager}}

\begin{fulllineitems}
\phantomsection\label{\detokenize{squad_state_manager:squad_state_manager.InterfazManager.on_arrow_left}}
\pysigstartsignatures
\pysiglinewithargsret{\sphinxbfcode{\sphinxupquote{on\_arrow\_left}}}{\sphinxparam{\DUrole{n,n}{event}}}{}
\pysigstopsignatures
\sphinxAtStartPar
Maneja la pulsación de la tecla de flecha hacia la izquierda para girar el robot a la izquierda.
\begin{quote}\begin{description}
\sphinxlineitem{Parámetros}
\sphinxAtStartPar
\sphinxstyleliteralstrong{\sphinxupquote{event}} \textendash{} Evento de la tecla pulsada.

\end{description}\end{quote}

\end{fulllineitems}

\index{on\_arrow\_right() (método de squad\_state\_manager.InterfazManager)@\spxentry{on\_arrow\_right()}\spxextra{método de squad\_state\_manager.InterfazManager}}

\begin{fulllineitems}
\phantomsection\label{\detokenize{squad_state_manager:squad_state_manager.InterfazManager.on_arrow_right}}
\pysigstartsignatures
\pysiglinewithargsret{\sphinxbfcode{\sphinxupquote{on\_arrow\_right}}}{\sphinxparam{\DUrole{n,n}{event}}}{}
\pysigstopsignatures
\sphinxAtStartPar
Maneja la pulsación de la tecla de flecha hacia la derecha para girar el robot a la derecha.
\begin{quote}\begin{description}
\sphinxlineitem{Parámetros}
\sphinxAtStartPar
\sphinxstyleliteralstrong{\sphinxupquote{event}} \textendash{} Evento de la tecla pulsada.

\end{description}\end{quote}

\end{fulllineitems}

\index{on\_arrow\_up() (método de squad\_state\_manager.InterfazManager)@\spxentry{on\_arrow\_up()}\spxextra{método de squad\_state\_manager.InterfazManager}}

\begin{fulllineitems}
\phantomsection\label{\detokenize{squad_state_manager:squad_state_manager.InterfazManager.on_arrow_up}}
\pysigstartsignatures
\pysiglinewithargsret{\sphinxbfcode{\sphinxupquote{on\_arrow\_up}}}{\sphinxparam{\DUrole{n,n}{event}}}{}
\pysigstopsignatures
\sphinxAtStartPar
Maneja la pulsación de la tecla de flecha hacia arriba para mover el robot hacia adelante.
\begin{quote}\begin{description}
\sphinxlineitem{Parámetros}
\sphinxAtStartPar
\sphinxstyleliteralstrong{\sphinxupquote{event}} \textendash{} Evento de la tecla pulsada.

\end{description}\end{quote}

\end{fulllineitems}

\index{on\_key\_release() (método de squad\_state\_manager.InterfazManager)@\spxentry{on\_key\_release()}\spxextra{método de squad\_state\_manager.InterfazManager}}

\begin{fulllineitems}
\phantomsection\label{\detokenize{squad_state_manager:squad_state_manager.InterfazManager.on_key_release}}
\pysigstartsignatures
\pysiglinewithargsret{\sphinxbfcode{\sphinxupquote{on\_key\_release}}}{\sphinxparam{\DUrole{n,n}{event}}}{}
\pysigstopsignatures
\sphinxAtStartPar
Maneja el evento de liberación de cualquier tecla para detener el movimiento del robot.
\begin{quote}\begin{description}
\sphinxlineitem{Parámetros}
\sphinxAtStartPar
\sphinxstyleliteralstrong{\sphinxupquote{event}} \textendash{} Evento de la tecla liberada.

\end{description}\end{quote}

\end{fulllineitems}

\index{on\_state\_button\_click() (método de squad\_state\_manager.InterfazManager)@\spxentry{on\_state\_button\_click()}\spxextra{método de squad\_state\_manager.InterfazManager}}

\begin{fulllineitems}
\phantomsection\label{\detokenize{squad_state_manager:squad_state_manager.InterfazManager.on_state_button_click}}
\pysigstartsignatures
\pysiglinewithargsret{\sphinxbfcode{\sphinxupquote{on\_state\_button\_click}}}{\sphinxparam{\DUrole{n,n}{command}}}{}
\pysigstopsignatures
\sphinxAtStartPar
Maneja el evento de clic en un botón de estado.

\sphinxAtStartPar
Publica el comando correspondiente al tópico “/command” para cambiar el estado
de la máquina de estados.
\begin{quote}\begin{description}
\sphinxlineitem{Parámetros}
\sphinxAtStartPar
\sphinxstyleliteralstrong{\sphinxupquote{command}} (\sphinxhref{https://docs.python.org/3/library/stdtypes.html\#str}{\sphinxstyleliteralemphasis{\sphinxupquote{str}}}) \textendash{} Comando a publicar.

\end{description}\end{quote}

\end{fulllineitems}

\index{setup\_gui() (método de squad\_state\_manager.InterfazManager)@\spxentry{setup\_gui()}\spxextra{método de squad\_state\_manager.InterfazManager}}

\begin{fulllineitems}
\phantomsection\label{\detokenize{squad_state_manager:squad_state_manager.InterfazManager.setup_gui}}
\pysigstartsignatures
\pysiglinewithargsret{\sphinxbfcode{\sphinxupquote{setup\_gui}}}{}{}
\pysigstopsignatures
\sphinxAtStartPar
Configura la interfaz gráfica de usuario utilizando Tkinter.

\sphinxAtStartPar
Crea la ventana principal, los botones de control, la visualización de la cámara
y los controles de exploración.

\end{fulllineitems}

\index{subscribe\_to\_camera() (método de squad\_state\_manager.InterfazManager)@\spxentry{subscribe\_to\_camera()}\spxextra{método de squad\_state\_manager.InterfazManager}}

\begin{fulllineitems}
\phantomsection\label{\detokenize{squad_state_manager:squad_state_manager.InterfazManager.subscribe_to_camera}}
\pysigstartsignatures
\pysiglinewithargsret{\sphinxbfcode{\sphinxupquote{subscribe\_to\_camera}}}{\sphinxparam{\DUrole{n,n}{topic}}}{}
\pysigstopsignatures
\sphinxAtStartPar
Suscribe al tópico de la cámara especificado.

\sphinxAtStartPar
Desuscribe del tópico de cámara anterior si existe y suscribe al nuevo tópico.
\begin{quote}\begin{description}
\sphinxlineitem{Parámetros}
\sphinxAtStartPar
\sphinxstyleliteralstrong{\sphinxupquote{topic}} (\sphinxhref{https://docs.python.org/3/library/stdtypes.html\#str}{\sphinxstyleliteralemphasis{\sphinxupquote{str}}}) \textendash{} Tópico de la cámara al que suscribirse.

\end{description}\end{quote}

\end{fulllineitems}

\index{switch\_camera\_topic() (método de squad\_state\_manager.InterfazManager)@\spxentry{switch\_camera\_topic()}\spxextra{método de squad\_state\_manager.InterfazManager}}

\begin{fulllineitems}
\phantomsection\label{\detokenize{squad_state_manager:squad_state_manager.InterfazManager.switch_camera_topic}}
\pysigstartsignatures
\pysiglinewithargsret{\sphinxbfcode{\sphinxupquote{switch\_camera\_topic}}}{\sphinxparam{\DUrole{n,n}{new\_topic}}}{}
\pysigstopsignatures
\sphinxAtStartPar
Cambia el tópico de la cámara a uno nuevo si es diferente al actual.
\begin{quote}\begin{description}
\sphinxlineitem{Parámetros}
\sphinxAtStartPar
\sphinxstyleliteralstrong{\sphinxupquote{new\_topic}} (\sphinxhref{https://docs.python.org/3/library/stdtypes.html\#str}{\sphinxstyleliteralemphasis{\sphinxupquote{str}}}) \textendash{} Nuevo tópico de la cámara al que suscribirse.

\end{description}\end{quote}

\end{fulllineitems}

\index{update\_state\_highlight() (método de squad\_state\_manager.InterfazManager)@\spxentry{update\_state\_highlight()}\spxextra{método de squad\_state\_manager.InterfazManager}}

\begin{fulllineitems}
\phantomsection\label{\detokenize{squad_state_manager:squad_state_manager.InterfazManager.update_state_highlight}}
\pysigstartsignatures
\pysiglinewithargsret{\sphinxbfcode{\sphinxupquote{update\_state\_highlight}}}{}{}
\pysigstopsignatures
\sphinxAtStartPar
Actualiza la interfaz gráfica para resaltar el botón correspondiente al estado actual
y mostrar u ocultar los controles de exploración según sea necesario.

\end{fulllineitems}


\end{fulllineitems}

\index{TurtleBotStateManager (clase en squad\_state\_manager)@\spxentry{TurtleBotStateManager}\spxextra{clase en squad\_state\_manager}}

\begin{fulllineitems}
\phantomsection\label{\detokenize{squad_state_manager:squad_state_manager.TurtleBotStateManager}}
\pysigstartsignatures
\pysigline{\sphinxbfcode{\sphinxupquote{class\DUrole{w,w}{  }}}\sphinxcode{\sphinxupquote{squad\_state\_manager.}}\sphinxbfcode{\sphinxupquote{TurtleBotStateManager}}}
\pysigstopsignatures
\sphinxAtStartPar
Bases: \sphinxhref{https://docs.python.org/3/library/functions.html\#object}{\sphinxcode{\sphinxupquote{object}}}

\sphinxAtStartPar
Gestor de Estados para el TurtleBot utilizando SMACH.

\sphinxAtStartPar
Esta clase inicializa la máquina de estados finita (SMACH) para gestionar los diferentes
comportamientos del TurtleBot, incluyendo reposo, exploración, navegación y aproximación a
objetivos. También configura la introspección para visualizar la máquina de estados y
lanza la interfaz gráfica para la interacción del usuario.
\begin{description}
\sphinxlineitem{Atributos:}
\sphinxAtStartPar
sm (StateMachine): Máquina de estados finita que define los diferentes estados y sus transiciones.
state\_machine\_thread (Thread): Hilo que ejecuta la máquina de estados.
gui (InterfazManager): Gestor de la interfaz gráfica de usuario.

\end{description}
\index{run\_state\_machine() (método de squad\_state\_manager.TurtleBotStateManager)@\spxentry{run\_state\_machine()}\spxextra{método de squad\_state\_manager.TurtleBotStateManager}}

\begin{fulllineitems}
\phantomsection\label{\detokenize{squad_state_manager:squad_state_manager.TurtleBotStateManager.run_state_machine}}
\pysigstartsignatures
\pysiglinewithargsret{\sphinxbfcode{\sphinxupquote{run\_state\_machine}}}{}{}
\pysigstopsignatures
\sphinxAtStartPar
Ejecuta la máquina de estados finita (SMACH).

\sphinxAtStartPar
Este método es ejecutado en un hilo separado para no bloquear el hilo principal.

\end{fulllineitems}


\end{fulllineitems}


\sphinxstepscope


\section{Paquete de Exploración}
\label{\detokenize{pkg_exploracion:paquete-de-exploracion}}\label{\detokenize{pkg_exploracion::doc}}
\sphinxAtStartPar
Este paquete contiene los servidores de acción necesarios para la detección y navegación.

\sphinxAtStartPar
\sphinxstylestrong{Módulos:}
\begin{itemize}
\item {} 
\sphinxAtStartPar
\sphinxstylestrong{squad\_object\_detection\_action.py:} Detección de objetos en imágenes RGBD.

\item {} 
\sphinxAtStartPar
\sphinxstylestrong{squad\_approach\_control\_action.py:} Aproximación a objetos detectados.

\item {} 
\sphinxAtStartPar
\sphinxstylestrong{squad\_autonomous\_control\_action.py:} Navegación autónoma evitando obstáculos.

\end{itemize}

\sphinxAtStartPar
\sphinxstylestrong{Nodos:}

\sphinxstepscope


\subsection{Detección de Objetos (squad\_object\_detection\_action)}
\label{\detokenize{squad_object_detection_action:deteccion-de-objetos-squad-object-detection-action}}\label{\detokenize{squad_object_detection_action::doc}}
\sphinxAtStartPar
Este nodo procesa imágenes RGB y de profundidad para detectar objetos en el entorno.

\sphinxAtStartPar
\sphinxstylestrong{Descripción del Nodo:}
\begin{itemize}
\item {} 
\sphinxAtStartPar
Convierte imágenes de ROS a OpenCV usando \sphinxtitleref{cv\_bridge}.

\item {} 
\sphinxAtStartPar
Detecta objetos por color (ejemplo: objetos rojos).

\item {} 
\sphinxAtStartPar
Calcula coordenadas globales utilizando datos de la cámara y la odometría.

\item {} 
\sphinxAtStartPar
Publica información en \sphinxtitleref{/detected\_objects}.

\end{itemize}

\sphinxAtStartPar
\sphinxstylestrong{Parámetros Importantes:}
\begin{itemize}
\item {} 
\sphinxAtStartPar
\sphinxstylestrong{\textasciigrave{}color\_threshold\textasciigrave{}}: Umbral de detección por color.

\item {} 
\sphinxAtStartPar
\sphinxstylestrong{\textasciigrave{}topic\_rgb\textasciigrave{}}: Tópico de entrada de imágenes RGB.

\item {} 
\sphinxAtStartPar
\sphinxstylestrong{\textasciigrave{}topic\_depth\textasciigrave{}}: Tópico de entrada de imágenes de profundidad.

\end{itemize}
\phantomsection\label{\detokenize{squad_object_detection_action:module-squad_object_detection_action}}\index{module@\spxentry{module}!squad\_object\_detection\_action@\spxentry{squad\_object\_detection\_action}}\index{squad\_object\_detection\_action@\spxentry{squad\_object\_detection\_action}!module@\spxentry{module}}
\sphinxAtStartPar
Servidor de Acción para la Detección de Objetos en TurtleBot.

\sphinxAtStartPar
Este nodo implementa un servidor de acción que permite al TurtleBot detectar objetos en su entorno
utilizando imágenes RGB y de profundidad. Coordina la conversión de imágenes ROS a OpenCV, la detección
de objetos basados en color, el cálculo de coordenadas mundiales de los objetos detectados y la publicación
de información relevante. Además, proporciona herramientas de depuración para visualizar imágenes
procesadas y datos en tiempo real.
\begin{description}
\sphinxlineitem{Funcionalidades Principales:}\begin{itemize}
\item {} 
\sphinxAtStartPar
Conversión de imágenes ROS a formatos utilizables por OpenCV.

\item {} 
\sphinxAtStartPar
Detección de objetos basados en color (por ejemplo, objetos rojos).

\item {} 
\sphinxAtStartPar
Cálculo de las coordenadas mundiales de los objetos detectados utilizando datos de
profundidad y odometría.

\item {} 
\sphinxAtStartPar
Publicación de información de objetos detectados en tópicos ROS.

\item {} 
\sphinxAtStartPar
Proporciona una interfaz de retroalimentación para la frecuencia de procesamiento.

\item {} 
\sphinxAtStartPar
Herramientas de depuración para visualizar imágenes procesadas y datos de profundidad.

\end{itemize}

\end{description}

\sphinxAtStartPar
Este servidor de acción está diseñado para ser parte integral de una máquina de estados finitos
(SMACH) que gestiona el comportamiento del robot en diferentes estados, como exploración,
aproximación a objetos y navegación.
\index{TurtleBotObjectDetectionAction (clase en squad\_object\_detection\_action)@\spxentry{TurtleBotObjectDetectionAction}\spxextra{clase en squad\_object\_detection\_action}}

\begin{fulllineitems}
\phantomsection\label{\detokenize{squad_object_detection_action:squad_object_detection_action.TurtleBotObjectDetectionAction}}
\pysigstartsignatures
\pysigline{\sphinxbfcode{\sphinxupquote{class\DUrole{w,w}{  }}}\sphinxcode{\sphinxupquote{squad\_object\_detection\_action.}}\sphinxbfcode{\sphinxupquote{TurtleBotObjectDetectionAction}}}
\pysigstopsignatures
\sphinxAtStartPar
Bases: \sphinxhref{https://docs.python.org/3/library/functions.html\#object}{\sphinxcode{\sphinxupquote{object}}}

\sphinxAtStartPar
Servidor de Acción para la Detección de Objetos en TurtleBot.

\sphinxAtStartPar
Este servidor de acción procesa imágenes RGB y de profundidad para detectar objetos de interés
en el entorno del TurtleBot. Coordina la conversión de imágenes, detección de objetos, cálculo
de coordenadas mundiales y publicación de información detectada.
\begin{description}
\sphinxlineitem{Hereda de:}
\sphinxAtStartPar
objectlib.SimpleActionServer: Proporciona funcionalidades de servidor de acción.

\sphinxlineitem{Atributos:}
\sphinxAtStartPar
server (SimpleActionServer): Servidor de acción para la detección de objetos.
bridge (CvBridge): Instancia de CvBridge para convertir imágenes ROS a OpenCV.
lock (threading.Lock): Bloqueo para controlar el acceso a datos compartidos.
object\_pub (Publisher): Publicador para la ubicación de objetos detectados.
process\_image\_pub (Publisher): Publicador para imágenes procesadas.
latest\_image (Image): Última imagen RGB recibida.
latest\_depth (Image): Última imagen de profundidad recibida.
latest\_position (PoseStamped): Última posición del robot.
latest\_depth\_cv (ndarray): Última imagen de profundidad convertida a OpenCV.
image\_count (int): Contador de imágenes procesadas.
processing\_times (deque): Cola para almacenar tiempos de procesamiento.
fx (float): Parámetro intrínseco de la cámara (focal length x).
fy (float): Parámetro intrínseco de la cámara (focal length y).
cx (float): Parámetro intrínseco de la cámara (principal point x).
cy (float): Parámetro intrínseco de la cámara (principal point y).
camera\_info\_sub (Subscriber): Suscriptor al tópico de CameraInfo.

\end{description}
\index{calculate\_world\_coordinates() (método de squad\_object\_detection\_action.TurtleBotObjectDetectionAction)@\spxentry{calculate\_world\_coordinates()}\spxextra{método de squad\_object\_detection\_action.TurtleBotObjectDetectionAction}}

\begin{fulllineitems}
\phantomsection\label{\detokenize{squad_object_detection_action:squad_object_detection_action.TurtleBotObjectDetectionAction.calculate_world_coordinates}}
\pysigstartsignatures
\pysiglinewithargsret{\sphinxbfcode{\sphinxupquote{calculate\_world\_coordinates}}}{\sphinxparam{\DUrole{n,n}{detected\_objects}}\sphinxparamcomma \sphinxparam{\DUrole{n,n}{processed\_image}}}{}
\pysigstopsignatures
\sphinxAtStartPar
Calcula las coordenadas mundiales de los objetos detectados a partir de sus coordenadas de píxel.

\sphinxAtStartPar
Utiliza la imagen de profundidad y los parámetros intrínsecos de la cámara para convertir
las coordenadas de píxel a coordenadas mundiales.
\begin{quote}\begin{description}
\sphinxlineitem{Parámetros}\begin{itemize}
\item {} 
\sphinxAtStartPar
\sphinxstyleliteralstrong{\sphinxupquote{detected\_objects}} (\sphinxhref{https://docs.python.org/3/library/stdtypes.html\#list}{\sphinxstyleliteralemphasis{\sphinxupquote{list}}}) \textendash{} Lista de objetos detectados con coordenadas de píxel.

\item {} 
\sphinxAtStartPar
\sphinxstyleliteralstrong{\sphinxupquote{processed\_image}} (\sphinxstyleliteralemphasis{\sphinxupquote{ndarray}}) \textendash{} Imagen procesada para dibujar información adicional.

\end{itemize}

\sphinxlineitem{Devuelve}
\sphinxAtStartPar
Lista de objetos con coordenadas mundiales y la imagen procesada con anotaciones.

\sphinxlineitem{Tipo del valor devuelto}
\sphinxAtStartPar
\sphinxhref{https://docs.python.org/3/library/stdtypes.html\#tuple}{tuple}

\end{description}\end{quote}

\end{fulllineitems}

\index{callback() (método de squad\_object\_detection\_action.TurtleBotObjectDetectionAction)@\spxentry{callback()}\spxextra{método de squad\_object\_detection\_action.TurtleBotObjectDetectionAction}}

\begin{fulllineitems}
\phantomsection\label{\detokenize{squad_object_detection_action:squad_object_detection_action.TurtleBotObjectDetectionAction.callback}}
\pysigstartsignatures
\pysiglinewithargsret{\sphinxbfcode{\sphinxupquote{callback}}}{\sphinxparam{\DUrole{n,n}{image\_msg}}\sphinxparamcomma \sphinxparam{\DUrole{n,n}{depth\_msg}}\sphinxparamcomma \sphinxparam{\DUrole{n,n}{odom\_msg}}}{}
\pysigstopsignatures
\sphinxAtStartPar
Callback para procesar los mensajes sincronizados de imagen, profundidad y odometría.

\sphinxAtStartPar
Convierte las imágenes de ROS a formatos utilizables por OpenCV y almacena los datos recibidos.
\begin{quote}\begin{description}
\sphinxlineitem{Parámetros}\begin{itemize}
\item {} 
\sphinxAtStartPar
\sphinxstyleliteralstrong{\sphinxupquote{image\_msg}} (\sphinxstyleliteralemphasis{\sphinxupquote{Image}}) \textendash{} Mensaje de imagen RGB.

\item {} 
\sphinxAtStartPar
\sphinxstyleliteralstrong{\sphinxupquote{depth\_msg}} (\sphinxstyleliteralemphasis{\sphinxupquote{Image}}) \textendash{} Mensaje de imagen de profundidad.

\item {} 
\sphinxAtStartPar
\sphinxstyleliteralstrong{\sphinxupquote{odom\_msg}} (\sphinxstyleliteralemphasis{\sphinxupquote{Odometry}}) \textendash{} Mensaje de odometría.

\end{itemize}

\end{description}\end{quote}

\end{fulllineitems}

\index{camera\_info\_callback() (método de squad\_object\_detection\_action.TurtleBotObjectDetectionAction)@\spxentry{camera\_info\_callback()}\spxextra{método de squad\_object\_detection\_action.TurtleBotObjectDetectionAction}}

\begin{fulllineitems}
\phantomsection\label{\detokenize{squad_object_detection_action:squad_object_detection_action.TurtleBotObjectDetectionAction.camera_info_callback}}
\pysigstartsignatures
\pysiglinewithargsret{\sphinxbfcode{\sphinxupquote{camera\_info\_callback}}}{\sphinxparam{\DUrole{n,n}{msg}}}{}
\pysigstopsignatures
\sphinxAtStartPar
Callback para procesar el mensaje de CameraInfo y extraer parámetros intrínsecos.

\sphinxAtStartPar
Almacena los parámetros intrínsecos de la cámara y desuscribe del tópico una vez recibidos.
\begin{quote}\begin{description}
\sphinxlineitem{Parámetros}
\sphinxAtStartPar
\sphinxstyleliteralstrong{\sphinxupquote{msg}} (\sphinxstyleliteralemphasis{\sphinxupquote{CameraInfo}}) \textendash{} Mensaje con información de la cámara.

\end{description}\end{quote}

\end{fulllineitems}

\index{detect\_objects() (método de squad\_object\_detection\_action.TurtleBotObjectDetectionAction)@\spxentry{detect\_objects()}\spxextra{método de squad\_object\_detection\_action.TurtleBotObjectDetectionAction}}

\begin{fulllineitems}
\phantomsection\label{\detokenize{squad_object_detection_action:squad_object_detection_action.TurtleBotObjectDetectionAction.detect_objects}}
\pysigstartsignatures
\pysiglinewithargsret{\sphinxbfcode{\sphinxupquote{detect\_objects}}}{\sphinxparam{\DUrole{n,n}{cv\_image}}}{}
\pysigstopsignatures
\sphinxAtStartPar
Detecta objetos de color específico en una imagen RGB.

\sphinxAtStartPar
Utiliza la conversión a espacio de color HSV y segmentación por color para identificar objetos
rojos, luego calcula sus coordenadas de píxeles.
\begin{quote}\begin{description}
\sphinxlineitem{Parámetros}
\sphinxAtStartPar
\sphinxstyleliteralstrong{\sphinxupquote{cv\_image}} (\sphinxstyleliteralemphasis{\sphinxupquote{ndarray}}) \textendash{} Imagen RGB en formato OpenCV.

\sphinxlineitem{Devuelve}
\sphinxAtStartPar
Imagen procesada y lista de objetos detectados con sus coordenadas de píxel.

\sphinxlineitem{Tipo del valor devuelto}
\sphinxAtStartPar
\sphinxhref{https://docs.python.org/3/library/stdtypes.html\#tuple}{tuple}

\end{description}\end{quote}

\end{fulllineitems}

\index{display\_debug\_info() (método de squad\_object\_detection\_action.TurtleBotObjectDetectionAction)@\spxentry{display\_debug\_info()}\spxextra{método de squad\_object\_detection\_action.TurtleBotObjectDetectionAction}}

\begin{fulllineitems}
\phantomsection\label{\detokenize{squad_object_detection_action:squad_object_detection_action.TurtleBotObjectDetectionAction.display_debug_info}}
\pysigstartsignatures
\pysiglinewithargsret{\sphinxbfcode{\sphinxupquote{display\_debug\_info}}}{\sphinxparam{\DUrole{n,n}{raw\_image}}\sphinxparamcomma \sphinxparam{\DUrole{n,n}{processed\_image}}\sphinxparamcomma \sphinxparam{\DUrole{n,n}{depth\_image}}}{}
\pysigstopsignatures
\sphinxAtStartPar
Muestra información de depuración en las imágenes procesadas.

\sphinxAtStartPar
Dibuja timestamps, posición del robot y profundidad sobre las imágenes.
\begin{quote}\begin{description}
\sphinxlineitem{Parámetros}\begin{itemize}
\item {} 
\sphinxAtStartPar
\sphinxstyleliteralstrong{\sphinxupquote{raw\_image}} (\sphinxstyleliteralemphasis{\sphinxupquote{ndarray}}) \textendash{} Imagen RGB original.

\item {} 
\sphinxAtStartPar
\sphinxstyleliteralstrong{\sphinxupquote{processed\_image}} (\sphinxstyleliteralemphasis{\sphinxupquote{ndarray}}) \textendash{} Imagen procesada con detecciones.

\item {} 
\sphinxAtStartPar
\sphinxstyleliteralstrong{\sphinxupquote{depth\_image}} (\sphinxstyleliteralemphasis{\sphinxupquote{ndarray}}) \textendash{} Imagen de profundidad procesada.

\end{itemize}

\end{description}\end{quote}

\end{fulllineitems}

\index{execute() (método de squad\_object\_detection\_action.TurtleBotObjectDetectionAction)@\spxentry{execute()}\spxextra{método de squad\_object\_detection\_action.TurtleBotObjectDetectionAction}}

\begin{fulllineitems}
\phantomsection\label{\detokenize{squad_object_detection_action:squad_object_detection_action.TurtleBotObjectDetectionAction.execute}}
\pysigstartsignatures
\pysiglinewithargsret{\sphinxbfcode{\sphinxupquote{execute}}}{\sphinxparam{\DUrole{n,n}{goal}}}{}
\pysigstopsignatures
\sphinxAtStartPar
Ejecuta la acción de detección de objetos.

\sphinxAtStartPar
Este método maneja el flujo principal de la acción, incluyendo la espera de CameraInfo,
procesamiento de imágenes y publicación de resultados.
\begin{quote}\begin{description}
\sphinxlineitem{Parámetros}
\sphinxAtStartPar
\sphinxstyleliteralstrong{\sphinxupquote{goal}} (\sphinxstyleliteralemphasis{\sphinxupquote{ObjectDetectionGoal}}) \textendash{} Objetivo de la acción recibido.

\end{description}\end{quote}

\end{fulllineitems}

\index{get\_depth\_at\_pixel() (método de squad\_object\_detection\_action.TurtleBotObjectDetectionAction)@\spxentry{get\_depth\_at\_pixel()}\spxextra{método de squad\_object\_detection\_action.TurtleBotObjectDetectionAction}}

\begin{fulllineitems}
\phantomsection\label{\detokenize{squad_object_detection_action:squad_object_detection_action.TurtleBotObjectDetectionAction.get_depth_at_pixel}}
\pysigstartsignatures
\pysiglinewithargsret{\sphinxbfcode{\sphinxupquote{get\_depth\_at\_pixel}}}{\sphinxparam{\DUrole{n,n}{x}}\sphinxparamcomma \sphinxparam{\DUrole{n,n}{y}}}{}
\pysigstopsignatures
\sphinxAtStartPar
Obtiene la profundidad en un píxel específico de la imagen de profundidad.

\sphinxAtStartPar
Verifica que las coordenadas estén dentro de los límites de la imagen y maneja valores inválidos.
\begin{quote}\begin{description}
\sphinxlineitem{Parámetros}\begin{itemize}
\item {} 
\sphinxAtStartPar
\sphinxstyleliteralstrong{\sphinxupquote{x}} (\sphinxhref{https://docs.python.org/3/library/functions.html\#int}{\sphinxstyleliteralemphasis{\sphinxupquote{int}}}) \textendash{} Coordenada x del píxel.

\item {} 
\sphinxAtStartPar
\sphinxstyleliteralstrong{\sphinxupquote{y}} (\sphinxhref{https://docs.python.org/3/library/functions.html\#int}{\sphinxstyleliteralemphasis{\sphinxupquote{int}}}) \textendash{} Coordenada y del píxel.

\end{itemize}

\sphinxlineitem{Devuelve}
\sphinxAtStartPar
Valor de profundidad en metros o None si es inválido.

\sphinxlineitem{Tipo del valor devuelto}
\sphinxAtStartPar
\sphinxhref{https://docs.python.org/3/library/functions.html\#float}{float} or None

\end{description}\end{quote}

\end{fulllineitems}

\index{load\_parameters() (método de squad\_object\_detection\_action.TurtleBotObjectDetectionAction)@\spxentry{load\_parameters()}\spxextra{método de squad\_object\_detection\_action.TurtleBotObjectDetectionAction}}

\begin{fulllineitems}
\phantomsection\label{\detokenize{squad_object_detection_action:squad_object_detection_action.TurtleBotObjectDetectionAction.load_parameters}}
\pysigstartsignatures
\pysiglinewithargsret{\sphinxbfcode{\sphinxupquote{load\_parameters}}}{}{}
\pysigstopsignatures
\sphinxAtStartPar
Carga los parámetros esenciales del nodo desde los archivos de configuración.

\sphinxAtStartPar
Establece los tópicos de suscripción, frecuencia de procesamiento y otros parámetros clave.

\end{fulllineitems}

\index{process\_latest\_data() (método de squad\_object\_detection\_action.TurtleBotObjectDetectionAction)@\spxentry{process\_latest\_data()}\spxextra{método de squad\_object\_detection\_action.TurtleBotObjectDetectionAction}}

\begin{fulllineitems}
\phantomsection\label{\detokenize{squad_object_detection_action:squad_object_detection_action.TurtleBotObjectDetectionAction.process_latest_data}}
\pysigstartsignatures
\pysiglinewithargsret{\sphinxbfcode{\sphinxupquote{process\_latest\_data}}}{\sphinxparam{\DUrole{n,n}{feedback}}}{}
\pysigstopsignatures
\sphinxAtStartPar
Procesa los datos más recientes de imagen, profundidad y posición.

\sphinxAtStartPar
Detecta objetos en la imagen, calcula sus coordenadas mundiales y publica la información.
\begin{quote}\begin{description}
\sphinxlineitem{Parámetros}
\sphinxAtStartPar
\sphinxstyleliteralstrong{\sphinxupquote{feedback}} (\sphinxstyleliteralemphasis{\sphinxupquote{ObjectDetectionFeedback}}) \textendash{} Feedback para actualizar la frecuencia de procesamiento.

\end{description}\end{quote}

\end{fulllineitems}

\index{subscribe\_to\_topics() (método de squad\_object\_detection\_action.TurtleBotObjectDetectionAction)@\spxentry{subscribe\_to\_topics()}\spxextra{método de squad\_object\_detection\_action.TurtleBotObjectDetectionAction}}

\begin{fulllineitems}
\phantomsection\label{\detokenize{squad_object_detection_action:squad_object_detection_action.TurtleBotObjectDetectionAction.subscribe_to_topics}}
\pysigstartsignatures
\pysiglinewithargsret{\sphinxbfcode{\sphinxupquote{subscribe\_to\_topics}}}{}{}
\pysigstopsignatures
\sphinxAtStartPar
Suscribe a los tópicos necesarios y sincroniza los mensajes recibidos.

\sphinxAtStartPar
Utiliza message\_filters para sincronizar las imágenes RGB, de profundidad y los datos de odometría.

\end{fulllineitems}

\index{transform\_camera\_to\_global() (método de squad\_object\_detection\_action.TurtleBotObjectDetectionAction)@\spxentry{transform\_camera\_to\_global()}\spxextra{método de squad\_object\_detection\_action.TurtleBotObjectDetectionAction}}

\begin{fulllineitems}
\phantomsection\label{\detokenize{squad_object_detection_action:squad_object_detection_action.TurtleBotObjectDetectionAction.transform_camera_to_global}}
\pysigstartsignatures
\pysiglinewithargsret{\sphinxbfcode{\sphinxupquote{transform\_camera\_to\_global}}}{\sphinxparam{\DUrole{n,n}{X\_cam}}\sphinxparamcomma \sphinxparam{\DUrole{n,n}{Y\_cam}}\sphinxparamcomma \sphinxparam{\DUrole{n,n}{Z\_cam}}}{}
\pysigstopsignatures
\sphinxAtStartPar
Transforma las coordenadas desde el marco de la cámara al marco global.

\sphinxAtStartPar
Utiliza la orientación y posición actual del robot para realizar la transformación.
\begin{quote}\begin{description}
\sphinxlineitem{Parámetros}\begin{itemize}
\item {} 
\sphinxAtStartPar
\sphinxstyleliteralstrong{\sphinxupquote{X\_cam}} (\sphinxhref{https://docs.python.org/3/library/functions.html\#float}{\sphinxstyleliteralemphasis{\sphinxupquote{float}}}) \textendash{} Coordenada X en el marco de la cámara.

\item {} 
\sphinxAtStartPar
\sphinxstyleliteralstrong{\sphinxupquote{Y\_cam}} (\sphinxhref{https://docs.python.org/3/library/functions.html\#float}{\sphinxstyleliteralemphasis{\sphinxupquote{float}}}) \textendash{} Coordenada Y en el marco de la cámara.

\item {} 
\sphinxAtStartPar
\sphinxstyleliteralstrong{\sphinxupquote{Z\_cam}} (\sphinxhref{https://docs.python.org/3/library/functions.html\#float}{\sphinxstyleliteralemphasis{\sphinxupquote{float}}}) \textendash{} Coordenada Z en el marco de la cámara.

\end{itemize}

\sphinxlineitem{Devuelve}
\sphinxAtStartPar
Coordenadas X y Y en el marco global o (None, None) si falla la transformación.

\sphinxlineitem{Tipo del valor devuelto}
\sphinxAtStartPar
\sphinxhref{https://docs.python.org/3/library/stdtypes.html\#tuple}{tuple}

\end{description}\end{quote}

\end{fulllineitems}


\end{fulllineitems}


\sphinxstepscope


\subsection{Aproximación a Objetos (squad\_approach\_control\_action)}
\label{\detokenize{squad_approach_control_action:aproximacion-a-objetos-squad-approach-control-action}}\label{\detokenize{squad_approach_control_action::doc}}
\sphinxAtStartPar
Este nodo mueve el robot hacia un objeto detectado.

\sphinxAtStartPar
\sphinxstylestrong{Descripción del Nodo:}
\begin{itemize}
\item {} 
\sphinxAtStartPar
Maneja una máquina de estados interna para controlar la aproximación.

\item {} 
\sphinxAtStartPar
Utiliza \sphinxtitleref{move\_base} para realizar movimientos precisos.

\item {} 
\sphinxAtStartPar
Configurable con parámetros como velocidades y distancia mínima.

\end{itemize}

\sphinxAtStartPar
\sphinxstylestrong{Parámetros Importantes:}
\begin{itemize}
\item {} 
\sphinxAtStartPar
\sphinxstylestrong{\textasciigrave{}linear\_speed\textasciigrave{}}: Velocidad lineal del robot.

\item {} 
\sphinxAtStartPar
\sphinxstylestrong{\textasciigrave{}angular\_speed\textasciigrave{}}: Velocidad angular del robot.

\item {} 
\sphinxAtStartPar
\sphinxstylestrong{\textasciigrave{}min\_distance\textasciigrave{}}: Distancia mínima al objetivo.

\end{itemize}
\phantomsection\label{\detokenize{squad_approach_control_action:module-squad_approach_control_action}}\index{module@\spxentry{module}!squad\_approach\_control\_action@\spxentry{squad\_approach\_control\_action}}\index{squad\_approach\_control\_action@\spxentry{squad\_approach\_control\_action}!module@\spxentry{module}}\index{ApproachObjectActionServer (clase en squad\_approach\_control\_action)@\spxentry{ApproachObjectActionServer}\spxextra{clase en squad\_approach\_control\_action}}

\begin{fulllineitems}
\phantomsection\label{\detokenize{squad_approach_control_action:squad_approach_control_action.ApproachObjectActionServer}}
\pysigstartsignatures
\pysiglinewithargsret{\sphinxbfcode{\sphinxupquote{class\DUrole{w,w}{  }}}\sphinxcode{\sphinxupquote{squad\_approach\_control\_action.}}\sphinxbfcode{\sphinxupquote{ApproachObjectActionServer}}}{\sphinxparam{\DUrole{n,n}{name}}}{}
\pysigstopsignatures
\sphinxAtStartPar
Bases: \sphinxhref{https://docs.python.org/3/library/functions.html\#object}{\sphinxcode{\sphinxupquote{object}}}

\sphinxAtStartPar
Servidor de acción para la aproximación al objeto detectado.

\sphinxAtStartPar
Esta clase implementa un servidor de acción que coordina la aproximación del robot hacia
un objeto detectado en el entorno. Utiliza una máquina de estados finitos (SMACH) para
gestionar las diferentes etapas de la aproximación, incluyendo la verificación de la posición,
movimiento hacia la posición objetivo, detección y centrado del objeto, aproximación final
y espera después de la aproximación. Además, maneja la preempción de la acción para permitir
la cancelación o interrupción del proceso en cualquier momento.
\index{execute\_cb() (método de squad\_approach\_control\_action.ApproachObjectActionServer)@\spxentry{execute\_cb()}\spxextra{método de squad\_approach\_control\_action.ApproachObjectActionServer}}

\begin{fulllineitems}
\phantomsection\label{\detokenize{squad_approach_control_action:squad_approach_control_action.ApproachObjectActionServer.execute_cb}}
\pysigstartsignatures
\pysiglinewithargsret{\sphinxbfcode{\sphinxupquote{execute\_cb}}}{\sphinxparam{\DUrole{n,n}{goal}}}{}
\pysigstopsignatures
\sphinxAtStartPar
Callback de ejecución de la acción.

\sphinxAtStartPar
Coordina la máquina de estados para aproximarse al objeto detectado. Gestiona las transiciones
entre estados como la verificación de la posición, movimiento hacia la posición objetivo,
detección y centrado del objeto, aproximación final y espera después de la aproximación.
\begin{quote}\begin{description}
\sphinxlineitem{Parámetros}
\sphinxAtStartPar
\sphinxstyleliteralstrong{\sphinxupquote{goal}} (\sphinxstyleliteralemphasis{\sphinxupquote{ApproachControlGoal}}) \textendash{} Objetivo de la acción (no utilizado en este caso).

\end{description}\end{quote}

\end{fulllineitems}

\index{object\_callback() (método de squad\_approach\_control\_action.ApproachObjectActionServer)@\spxentry{object\_callback()}\spxextra{método de squad\_approach\_control\_action.ApproachObjectActionServer}}

\begin{fulllineitems}
\phantomsection\label{\detokenize{squad_approach_control_action:squad_approach_control_action.ApproachObjectActionServer.object_callback}}
\pysigstartsignatures
\pysiglinewithargsret{\sphinxbfcode{\sphinxupquote{object\_callback}}}{\sphinxparam{\DUrole{n,n}{msg}}}{}
\pysigstopsignatures
\sphinxAtStartPar
Callback para almacenar la información del objeto detectado.

\sphinxAtStartPar
Actualiza el objeto detectado con los datos recibidos del tópico de detección de objetos.
\begin{quote}\begin{description}
\sphinxlineitem{Parámetros}
\sphinxAtStartPar
\sphinxstyleliteralstrong{\sphinxupquote{msg}} (\sphinxstyleliteralemphasis{\sphinxupquote{DetectedObject}}) \textendash{} Mensaje con la información del objeto detectado.

\end{description}\end{quote}

\end{fulllineitems}

\index{odom\_callback() (método de squad\_approach\_control\_action.ApproachObjectActionServer)@\spxentry{odom\_callback()}\spxextra{método de squad\_approach\_control\_action.ApproachObjectActionServer}}

\begin{fulllineitems}
\phantomsection\label{\detokenize{squad_approach_control_action:squad_approach_control_action.ApproachObjectActionServer.odom_callback}}
\pysigstartsignatures
\pysiglinewithargsret{\sphinxbfcode{\sphinxupquote{odom\_callback}}}{\sphinxparam{\DUrole{n,n}{msg}}}{}
\pysigstopsignatures
\sphinxAtStartPar
Callback para almacenar los datos de odometría.

\sphinxAtStartPar
Actualiza la posición actual del robot con los datos recibidos del tópico de odometría.
\begin{quote}\begin{description}
\sphinxlineitem{Parámetros}
\sphinxAtStartPar
\sphinxstyleliteralstrong{\sphinxupquote{msg}} (\sphinxstyleliteralemphasis{\sphinxupquote{Odometry}}) \textendash{} Mensaje con la información de la posición del robot.

\end{description}\end{quote}

\end{fulllineitems}

\index{preempt\_cb() (método de squad\_approach\_control\_action.ApproachObjectActionServer)@\spxentry{preempt\_cb()}\spxextra{método de squad\_approach\_control\_action.ApproachObjectActionServer}}

\begin{fulllineitems}
\phantomsection\label{\detokenize{squad_approach_control_action:squad_approach_control_action.ApproachObjectActionServer.preempt_cb}}
\pysigstartsignatures
\pysiglinewithargsret{\sphinxbfcode{\sphinxupquote{preempt\_cb}}}{}{}
\pysigstopsignatures
\sphinxAtStartPar
Callback para manejar la preempción de la acción.

\sphinxAtStartPar
Establece la bandera de preempción para indicar que se ha solicitado la interrupción
de la acción en curso.

\end{fulllineitems}


\end{fulllineitems}

\index{ApproachObjectState (clase en squad\_approach\_control\_action)@\spxentry{ApproachObjectState}\spxextra{clase en squad\_approach\_control\_action}}

\begin{fulllineitems}
\phantomsection\label{\detokenize{squad_approach_control_action:squad_approach_control_action.ApproachObjectState}}
\pysigstartsignatures
\pysiglinewithargsret{\sphinxbfcode{\sphinxupquote{class\DUrole{w,w}{  }}}\sphinxcode{\sphinxupquote{squad\_approach\_control\_action.}}\sphinxbfcode{\sphinxupquote{ApproachObjectState}}}{\sphinxparam{\DUrole{o,o}{*}\DUrole{n,n}{args}\DUrole{p,p}{:}\DUrole{w,w}{  }\DUrole{n,n}{\sphinxhref{https://docs.python.org/3/library/typing.html\#typing.Any}{Any}}}\sphinxparamcomma \sphinxparam{\DUrole{o,o}{**}\DUrole{n,n}{kwargs}\DUrole{p,p}{:}\DUrole{w,w}{  }\DUrole{n,n}{\sphinxhref{https://docs.python.org/3/library/typing.html\#typing.Any}{Any}}}}{}
\pysigstopsignatures
\sphinxAtStartPar
Bases: \sphinxcode{\sphinxupquote{State}}

\sphinxAtStartPar
Estado para aproximarse al objeto detectado.

\sphinxAtStartPar
Este estado mueve el robot hacia adelante en línea recta hasta que alcanza una distancia
mínima especificada del objeto detectado. Monitorea continuamente la distancia al objeto
para determinar cuándo detenerse.
\index{execute() (método de squad\_approach\_control\_action.ApproachObjectState)@\spxentry{execute()}\spxextra{método de squad\_approach\_control\_action.ApproachObjectState}}

\begin{fulllineitems}
\phantomsection\label{\detokenize{squad_approach_control_action:squad_approach_control_action.ApproachObjectState.execute}}
\pysigstartsignatures
\pysiglinewithargsret{\sphinxbfcode{\sphinxupquote{execute}}}{\sphinxparam{\DUrole{n,n}{userdata}}}{}
\pysigstopsignatures
\sphinxAtStartPar
Ejecuta la lógica del estado para aproximarse al objeto detectado.

\sphinxAtStartPar
Mueve el robot hacia adelante hasta que la distancia al objeto detectado sea menor o
igual a la distancia mínima especificada. Si la distancia es adecuada, detiene el robot.
\begin{quote}\begin{description}
\sphinxlineitem{Parámetros}
\sphinxAtStartPar
\sphinxstyleliteralstrong{\sphinxupquote{userdata}} (\sphinxstyleliteralemphasis{\sphinxupquote{smach.UserData}}) \textendash{} Datos de usuario proporcionados por SMACH.

\sphinxlineitem{Devuelve}
\sphinxAtStartPar
Outcome del estado.

\sphinxlineitem{Tipo del valor devuelto}
\sphinxAtStartPar
\sphinxhref{https://docs.python.org/3/library/stdtypes.html\#str}{str}

\end{description}\end{quote}

\end{fulllineitems}

\index{object\_callback() (método de squad\_approach\_control\_action.ApproachObjectState)@\spxentry{object\_callback()}\spxextra{método de squad\_approach\_control\_action.ApproachObjectState}}

\begin{fulllineitems}
\phantomsection\label{\detokenize{squad_approach_control_action:squad_approach_control_action.ApproachObjectState.object_callback}}
\pysigstartsignatures
\pysiglinewithargsret{\sphinxbfcode{\sphinxupquote{object\_callback}}}{\sphinxparam{\DUrole{n,n}{msg}}}{}
\pysigstopsignatures
\sphinxAtStartPar
Callback para almacenar la información del objeto detectado.

\sphinxAtStartPar
Actualiza el objeto detectado con los datos recibidos del tópico de detección de objetos.
\begin{quote}\begin{description}
\sphinxlineitem{Parámetros}
\sphinxAtStartPar
\sphinxstyleliteralstrong{\sphinxupquote{msg}} (\sphinxstyleliteralemphasis{\sphinxupquote{DetectedObject}}) \textendash{} Mensaje con la información del objeto detectado.

\end{description}\end{quote}

\end{fulllineitems}


\end{fulllineitems}

\index{CenterObjectState (clase en squad\_approach\_control\_action)@\spxentry{CenterObjectState}\spxextra{clase en squad\_approach\_control\_action}}

\begin{fulllineitems}
\phantomsection\label{\detokenize{squad_approach_control_action:squad_approach_control_action.CenterObjectState}}
\pysigstartsignatures
\pysiglinewithargsret{\sphinxbfcode{\sphinxupquote{class\DUrole{w,w}{  }}}\sphinxcode{\sphinxupquote{squad\_approach\_control\_action.}}\sphinxbfcode{\sphinxupquote{CenterObjectState}}}{\sphinxparam{\DUrole{o,o}{*}\DUrole{n,n}{args}\DUrole{p,p}{:}\DUrole{w,w}{  }\DUrole{n,n}{\sphinxhref{https://docs.python.org/3/library/typing.html\#typing.Any}{Any}}}\sphinxparamcomma \sphinxparam{\DUrole{o,o}{**}\DUrole{n,n}{kwargs}\DUrole{p,p}{:}\DUrole{w,w}{  }\DUrole{n,n}{\sphinxhref{https://docs.python.org/3/library/typing.html\#typing.Any}{Any}}}}{}
\pysigstopsignatures
\sphinxAtStartPar
Bases: \sphinxcode{\sphinxupquote{State}}

\sphinxAtStartPar
Estado para centrar el objeto detectado en el campo de visión de la cámara.

\sphinxAtStartPar
Este estado ajusta la orientación del robot girando hacia el objeto detectado hasta que
esté centrado en la imagen de la cámara. Utiliza las coordenadas del objeto para determinar
la dirección del giro.
\index{execute() (método de squad\_approach\_control\_action.CenterObjectState)@\spxentry{execute()}\spxextra{método de squad\_approach\_control\_action.CenterObjectState}}

\begin{fulllineitems}
\phantomsection\label{\detokenize{squad_approach_control_action:squad_approach_control_action.CenterObjectState.execute}}
\pysigstartsignatures
\pysiglinewithargsret{\sphinxbfcode{\sphinxupquote{execute}}}{\sphinxparam{\DUrole{n,n}{userdata}}}{}
\pysigstopsignatures
\sphinxAtStartPar
Ejecuta la lógica del estado para centrar el objeto en la cámara.

\sphinxAtStartPar
Ajusta la orientación del robot girando hacia el objeto detectado hasta que esté
centrado en el campo de visión de la cámara. Si el objeto no está centrado dentro
de una tolerancia definida, continúa girando en la dirección apropiada.
\begin{quote}\begin{description}
\sphinxlineitem{Parámetros}
\sphinxAtStartPar
\sphinxstyleliteralstrong{\sphinxupquote{userdata}} (\sphinxstyleliteralemphasis{\sphinxupquote{smach.UserData}}) \textendash{} Datos de usuario proporcionados por SMACH.

\sphinxlineitem{Devuelve}
\sphinxAtStartPar
Outcome del estado.

\sphinxlineitem{Tipo del valor devuelto}
\sphinxAtStartPar
\sphinxhref{https://docs.python.org/3/library/stdtypes.html\#str}{str}

\end{description}\end{quote}

\end{fulllineitems}

\index{object\_callback() (método de squad\_approach\_control\_action.CenterObjectState)@\spxentry{object\_callback()}\spxextra{método de squad\_approach\_control\_action.CenterObjectState}}

\begin{fulllineitems}
\phantomsection\label{\detokenize{squad_approach_control_action:squad_approach_control_action.CenterObjectState.object_callback}}
\pysigstartsignatures
\pysiglinewithargsret{\sphinxbfcode{\sphinxupquote{object\_callback}}}{\sphinxparam{\DUrole{n,n}{msg}}}{}
\pysigstopsignatures
\sphinxAtStartPar
Callback para almacenar la información del objeto detectado.

\sphinxAtStartPar
Actualiza el objeto detectado con los datos recibidos del tópico de detección de objetos.
\begin{quote}\begin{description}
\sphinxlineitem{Parámetros}
\sphinxAtStartPar
\sphinxstyleliteralstrong{\sphinxupquote{msg}} (\sphinxstyleliteralemphasis{\sphinxupquote{DetectedObject}}) \textendash{} Mensaje con la información del objeto detectado.

\end{description}\end{quote}

\end{fulllineitems}


\end{fulllineitems}

\index{CheckPositionState (clase en squad\_approach\_control\_action)@\spxentry{CheckPositionState}\spxextra{clase en squad\_approach\_control\_action}}

\begin{fulllineitems}
\phantomsection\label{\detokenize{squad_approach_control_action:squad_approach_control_action.CheckPositionState}}
\pysigstartsignatures
\pysiglinewithargsret{\sphinxbfcode{\sphinxupquote{class\DUrole{w,w}{  }}}\sphinxcode{\sphinxupquote{squad\_approach\_control\_action.}}\sphinxbfcode{\sphinxupquote{CheckPositionState}}}{\sphinxparam{\DUrole{o,o}{*}\DUrole{n,n}{args}\DUrole{p,p}{:}\DUrole{w,w}{  }\DUrole{n,n}{\sphinxhref{https://docs.python.org/3/library/typing.html\#typing.Any}{Any}}}\sphinxparamcomma \sphinxparam{\DUrole{o,o}{**}\DUrole{n,n}{kwargs}\DUrole{p,p}{:}\DUrole{w,w}{  }\DUrole{n,n}{\sphinxhref{https://docs.python.org/3/library/typing.html\#typing.Any}{Any}}}}{}
\pysigstopsignatures
\sphinxAtStartPar
Bases: \sphinxcode{\sphinxupquote{State}}

\sphinxAtStartPar
Estado para verificar la posición actual del robot respecto al objetivo.

\sphinxAtStartPar
Este estado calcula la distancia y la diferencia de orientación entre la posición actual
del robot y la posición objetivo. Determina si el robot ha alcanzado la posición deseada
dentro de una tolerancia definida.
\index{calculate\_distance() (método de squad\_approach\_control\_action.CheckPositionState)@\spxentry{calculate\_distance()}\spxextra{método de squad\_approach\_control\_action.CheckPositionState}}

\begin{fulllineitems}
\phantomsection\label{\detokenize{squad_approach_control_action:squad_approach_control_action.CheckPositionState.calculate_distance}}
\pysigstartsignatures
\pysiglinewithargsret{\sphinxbfcode{\sphinxupquote{calculate\_distance}}}{\sphinxparam{\DUrole{n,n}{pos1}}\sphinxparamcomma \sphinxparam{\DUrole{n,n}{pos2}}}{}
\pysigstopsignatures
\sphinxAtStartPar
Calcula la distancia euclidiana entre dos posiciones.
\begin{quote}\begin{description}
\sphinxlineitem{Parámetros}\begin{itemize}
\item {} 
\sphinxAtStartPar
\sphinxstyleliteralstrong{\sphinxupquote{pos1}} (\sphinxstyleliteralemphasis{\sphinxupquote{geometry\_msgs/Point}}) \textendash{} Primera posición.

\item {} 
\sphinxAtStartPar
\sphinxstyleliteralstrong{\sphinxupquote{pos2}} (\sphinxstyleliteralemphasis{\sphinxupquote{geometry\_msgs/Point}}) \textendash{} Segunda posición.

\end{itemize}

\sphinxlineitem{Devuelve}
\sphinxAtStartPar
Distancia en metros.

\sphinxlineitem{Tipo del valor devuelto}
\sphinxAtStartPar
\sphinxhref{https://docs.python.org/3/library/functions.html\#float}{float}

\end{description}\end{quote}

\end{fulllineitems}

\index{calculate\_orientation\_diff() (método de squad\_approach\_control\_action.CheckPositionState)@\spxentry{calculate\_orientation\_diff()}\spxextra{método de squad\_approach\_control\_action.CheckPositionState}}

\begin{fulllineitems}
\phantomsection\label{\detokenize{squad_approach_control_action:squad_approach_control_action.CheckPositionState.calculate_orientation_diff}}
\pysigstartsignatures
\pysiglinewithargsret{\sphinxbfcode{\sphinxupquote{calculate\_orientation\_diff}}}{\sphinxparam{\DUrole{n,n}{ori1}}\sphinxparamcomma \sphinxparam{\DUrole{n,n}{ori2}}}{}
\pysigstopsignatures
\sphinxAtStartPar
Calcula la diferencia de orientación en yaw entre dos cuaterniones.
\begin{quote}\begin{description}
\sphinxlineitem{Parámetros}\begin{itemize}
\item {} 
\sphinxAtStartPar
\sphinxstyleliteralstrong{\sphinxupquote{ori1}} (\sphinxstyleliteralemphasis{\sphinxupquote{geometry\_msgs/Quaternion}}) \textendash{} Primera orientación.

\item {} 
\sphinxAtStartPar
\sphinxstyleliteralstrong{\sphinxupquote{ori2}} (\sphinxstyleliteralemphasis{\sphinxupquote{geometry\_msgs/Quaternion}}) \textendash{} Segunda orientación.

\end{itemize}

\sphinxlineitem{Devuelve}
\sphinxAtStartPar
Diferencia de orientación en radianes.

\sphinxlineitem{Tipo del valor devuelto}
\sphinxAtStartPar
\sphinxhref{https://docs.python.org/3/library/functions.html\#float}{float}

\end{description}\end{quote}

\end{fulllineitems}

\index{execute() (método de squad\_approach\_control\_action.CheckPositionState)@\spxentry{execute()}\spxextra{método de squad\_approach\_control\_action.CheckPositionState}}

\begin{fulllineitems}
\phantomsection\label{\detokenize{squad_approach_control_action:squad_approach_control_action.CheckPositionState.execute}}
\pysigstartsignatures
\pysiglinewithargsret{\sphinxbfcode{\sphinxupquote{execute}}}{\sphinxparam{\DUrole{n,n}{userdata}}}{}
\pysigstopsignatures
\sphinxAtStartPar
Ejecuta la lógica del estado para verificar la posición.

\sphinxAtStartPar
Calcula la distancia y la diferencia de orientación entre la posición actual y el objetivo.
Determina si el robot está dentro de las tolerancias definidas para considerar que ha
alcanzado la posición objetivo.
\begin{quote}\begin{description}
\sphinxlineitem{Parámetros}
\sphinxAtStartPar
\sphinxstyleliteralstrong{\sphinxupquote{userdata}} (\sphinxstyleliteralemphasis{\sphinxupquote{smach.UserData}}) \textendash{} Datos de usuario proporcionados por SMACH.

\sphinxlineitem{Devuelve}
\sphinxAtStartPar
Outcome del estado.

\sphinxlineitem{Tipo del valor devuelto}
\sphinxAtStartPar
\sphinxhref{https://docs.python.org/3/library/stdtypes.html\#str}{str}

\end{description}\end{quote}

\end{fulllineitems}

\index{quaternion\_to\_yaw() (método de squad\_approach\_control\_action.CheckPositionState)@\spxentry{quaternion\_to\_yaw()}\spxextra{método de squad\_approach\_control\_action.CheckPositionState}}

\begin{fulllineitems}
\phantomsection\label{\detokenize{squad_approach_control_action:squad_approach_control_action.CheckPositionState.quaternion_to_yaw}}
\pysigstartsignatures
\pysiglinewithargsret{\sphinxbfcode{\sphinxupquote{quaternion\_to\_yaw}}}{\sphinxparam{\DUrole{n,n}{quat}}}{}
\pysigstopsignatures
\sphinxAtStartPar
Convierte un cuaternión a un ángulo yaw.
\begin{quote}\begin{description}
\sphinxlineitem{Parámetros}
\sphinxAtStartPar
\sphinxstyleliteralstrong{\sphinxupquote{quat}} (\sphinxstyleliteralemphasis{\sphinxupquote{geometry\_msgs/Quaternion}}) \textendash{} Cuaternión.

\sphinxlineitem{Devuelve}
\sphinxAtStartPar
Ángulo yaw en radianes.

\sphinxlineitem{Tipo del valor devuelto}
\sphinxAtStartPar
\sphinxhref{https://docs.python.org/3/library/functions.html\#float}{float}

\end{description}\end{quote}

\end{fulllineitems}


\end{fulllineitems}

\index{MoveToPositionState (clase en squad\_approach\_control\_action)@\spxentry{MoveToPositionState}\spxextra{clase en squad\_approach\_control\_action}}

\begin{fulllineitems}
\phantomsection\label{\detokenize{squad_approach_control_action:squad_approach_control_action.MoveToPositionState}}
\pysigstartsignatures
\pysiglinewithargsret{\sphinxbfcode{\sphinxupquote{class\DUrole{w,w}{  }}}\sphinxcode{\sphinxupquote{squad\_approach\_control\_action.}}\sphinxbfcode{\sphinxupquote{MoveToPositionState}}}{\sphinxparam{\DUrole{o,o}{*}\DUrole{n,n}{args}\DUrole{p,p}{:}\DUrole{w,w}{  }\DUrole{n,n}{\sphinxhref{https://docs.python.org/3/library/typing.html\#typing.Any}{Any}}}\sphinxparamcomma \sphinxparam{\DUrole{o,o}{**}\DUrole{n,n}{kwargs}\DUrole{p,p}{:}\DUrole{w,w}{  }\DUrole{n,n}{\sphinxhref{https://docs.python.org/3/library/typing.html\#typing.Any}{Any}}}}{}
\pysigstopsignatures
\sphinxAtStartPar
Bases: \sphinxcode{\sphinxupquote{State}}

\sphinxAtStartPar
Estado para mover el robot hacia la posición objetivo utilizando move\_base.

\sphinxAtStartPar
Este estado envía un objetivo al servidor move\_base para que el robot se desplace
hacia la posición deseada. Monitorea el estado de la acción y maneja posibles
resultados como éxito, aborto o preempción.
\index{execute() (método de squad\_approach\_control\_action.MoveToPositionState)@\spxentry{execute()}\spxextra{método de squad\_approach\_control\_action.MoveToPositionState}}

\begin{fulllineitems}
\phantomsection\label{\detokenize{squad_approach_control_action:squad_approach_control_action.MoveToPositionState.execute}}
\pysigstartsignatures
\pysiglinewithargsret{\sphinxbfcode{\sphinxupquote{execute}}}{\sphinxparam{\DUrole{n,n}{userdata}}}{}
\pysigstopsignatures
\sphinxAtStartPar
Ejecuta la lógica del estado para mover el robot hacia la posición objetivo.

\sphinxAtStartPar
Envía un objetivo al servidor move\_base y monitorea su estado hasta que se
complete la acción o se solicite una preempción.
\begin{quote}\begin{description}
\sphinxlineitem{Parámetros}
\sphinxAtStartPar
\sphinxstyleliteralstrong{\sphinxupquote{userdata}} (\sphinxstyleliteralemphasis{\sphinxupquote{smach.UserData}}) \textendash{} Datos de usuario proporcionados por SMACH.

\sphinxlineitem{Devuelve}
\sphinxAtStartPar
Outcome del estado.

\sphinxlineitem{Tipo del valor devuelto}
\sphinxAtStartPar
\sphinxhref{https://docs.python.org/3/library/stdtypes.html\#str}{str}

\end{description}\end{quote}

\end{fulllineitems}


\end{fulllineitems}

\index{StartObjectDetectionState (clase en squad\_approach\_control\_action)@\spxentry{StartObjectDetectionState}\spxextra{clase en squad\_approach\_control\_action}}

\begin{fulllineitems}
\phantomsection\label{\detokenize{squad_approach_control_action:squad_approach_control_action.StartObjectDetectionState}}
\pysigstartsignatures
\pysiglinewithargsret{\sphinxbfcode{\sphinxupquote{class\DUrole{w,w}{  }}}\sphinxcode{\sphinxupquote{squad\_approach\_control\_action.}}\sphinxbfcode{\sphinxupquote{StartObjectDetectionState}}}{\sphinxparam{\DUrole{o,o}{*}\DUrole{n,n}{args}\DUrole{p,p}{:}\DUrole{w,w}{  }\DUrole{n,n}{\sphinxhref{https://docs.python.org/3/library/typing.html\#typing.Any}{Any}}}\sphinxparamcomma \sphinxparam{\DUrole{o,o}{**}\DUrole{n,n}{kwargs}\DUrole{p,p}{:}\DUrole{w,w}{  }\DUrole{n,n}{\sphinxhref{https://docs.python.org/3/library/typing.html\#typing.Any}{Any}}}}{}
\pysigstopsignatures
\sphinxAtStartPar
Bases: \sphinxcode{\sphinxupquote{State}}

\sphinxAtStartPar
Estado para iniciar la detección de objetos mediante el servidor de acción de detección.

\sphinxAtStartPar
Este estado envía un objetivo al servidor de acción de detección de objetos para comenzar
la detección. Monitorea el estado de la acción y maneja posibles resultados como éxito,
aborto o preempción.
\index{execute() (método de squad\_approach\_control\_action.StartObjectDetectionState)@\spxentry{execute()}\spxextra{método de squad\_approach\_control\_action.StartObjectDetectionState}}

\begin{fulllineitems}
\phantomsection\label{\detokenize{squad_approach_control_action:squad_approach_control_action.StartObjectDetectionState.execute}}
\pysigstartsignatures
\pysiglinewithargsret{\sphinxbfcode{\sphinxupquote{execute}}}{\sphinxparam{\DUrole{n,n}{userdata}}}{}
\pysigstopsignatures
\sphinxAtStartPar
Ejecuta la lógica del estado para iniciar la detección de objetos.

\sphinxAtStartPar
Envía un objetivo vacío al servidor de acción de detección de objetos para comenzar
la detección. Verifica si el servidor está disponible y maneja posibles preempciones.
\begin{quote}\begin{description}
\sphinxlineitem{Parámetros}
\sphinxAtStartPar
\sphinxstyleliteralstrong{\sphinxupquote{userdata}} (\sphinxstyleliteralemphasis{\sphinxupquote{smach.UserData}}) \textendash{} Datos de usuario proporcionados por SMACH.

\sphinxlineitem{Devuelve}
\sphinxAtStartPar
Outcome del estado.

\sphinxlineitem{Tipo del valor devuelto}
\sphinxAtStartPar
\sphinxhref{https://docs.python.org/3/library/stdtypes.html\#str}{str}

\end{description}\end{quote}

\end{fulllineitems}


\end{fulllineitems}

\index{StopObjectDetectionState (clase en squad\_approach\_control\_action)@\spxentry{StopObjectDetectionState}\spxextra{clase en squad\_approach\_control\_action}}

\begin{fulllineitems}
\phantomsection\label{\detokenize{squad_approach_control_action:squad_approach_control_action.StopObjectDetectionState}}
\pysigstartsignatures
\pysiglinewithargsret{\sphinxbfcode{\sphinxupquote{class\DUrole{w,w}{  }}}\sphinxcode{\sphinxupquote{squad\_approach\_control\_action.}}\sphinxbfcode{\sphinxupquote{StopObjectDetectionState}}}{\sphinxparam{\DUrole{o,o}{*}\DUrole{n,n}{args}\DUrole{p,p}{:}\DUrole{w,w}{  }\DUrole{n,n}{\sphinxhref{https://docs.python.org/3/library/typing.html\#typing.Any}{Any}}}\sphinxparamcomma \sphinxparam{\DUrole{o,o}{**}\DUrole{n,n}{kwargs}\DUrole{p,p}{:}\DUrole{w,w}{  }\DUrole{n,n}{\sphinxhref{https://docs.python.org/3/library/typing.html\#typing.Any}{Any}}}}{}
\pysigstopsignatures
\sphinxAtStartPar
Bases: \sphinxcode{\sphinxupquote{State}}

\sphinxAtStartPar
Estado para detener la detección de objetos mediante el servidor de acción de detección.

\sphinxAtStartPar
Este estado cancela cualquier objetivo pendiente o en curso en el servidor de acción
de detección de objetos para detener la detección.
\index{execute() (método de squad\_approach\_control\_action.StopObjectDetectionState)@\spxentry{execute()}\spxextra{método de squad\_approach\_control\_action.StopObjectDetectionState}}

\begin{fulllineitems}
\phantomsection\label{\detokenize{squad_approach_control_action:squad_approach_control_action.StopObjectDetectionState.execute}}
\pysigstartsignatures
\pysiglinewithargsret{\sphinxbfcode{\sphinxupquote{execute}}}{\sphinxparam{\DUrole{n,n}{userdata}}}{}
\pysigstopsignatures
\sphinxAtStartPar
Ejecuta la lógica del estado para detener la detección de objetos.

\sphinxAtStartPar
Cancela cualquier objetivo activo o pendiente en el servidor de acción de detección
de objetos y maneja posibles preempciones.
\begin{quote}\begin{description}
\sphinxlineitem{Parámetros}
\sphinxAtStartPar
\sphinxstyleliteralstrong{\sphinxupquote{userdata}} (\sphinxstyleliteralemphasis{\sphinxupquote{smach.UserData}}) \textendash{} Datos de usuario proporcionados por SMACH.

\sphinxlineitem{Devuelve}
\sphinxAtStartPar
Outcome del estado.

\sphinxlineitem{Tipo del valor devuelto}
\sphinxAtStartPar
\sphinxhref{https://docs.python.org/3/library/stdtypes.html\#str}{str}

\end{description}\end{quote}

\end{fulllineitems}


\end{fulllineitems}

\index{WaitState (clase en squad\_approach\_control\_action)@\spxentry{WaitState}\spxextra{clase en squad\_approach\_control\_action}}

\begin{fulllineitems}
\phantomsection\label{\detokenize{squad_approach_control_action:squad_approach_control_action.WaitState}}
\pysigstartsignatures
\pysiglinewithargsret{\sphinxbfcode{\sphinxupquote{class\DUrole{w,w}{  }}}\sphinxcode{\sphinxupquote{squad\_approach\_control\_action.}}\sphinxbfcode{\sphinxupquote{WaitState}}}{\sphinxparam{\DUrole{o,o}{*}\DUrole{n,n}{args}\DUrole{p,p}{:}\DUrole{w,w}{  }\DUrole{n,n}{\sphinxhref{https://docs.python.org/3/library/typing.html\#typing.Any}{Any}}}\sphinxparamcomma \sphinxparam{\DUrole{o,o}{**}\DUrole{n,n}{kwargs}\DUrole{p,p}{:}\DUrole{w,w}{  }\DUrole{n,n}{\sphinxhref{https://docs.python.org/3/library/typing.html\#typing.Any}{Any}}}}{}
\pysigstopsignatures
\sphinxAtStartPar
Bases: \sphinxcode{\sphinxupquote{State}}

\sphinxAtStartPar
Estado de espera tras aproximarse al objeto.

\sphinxAtStartPar
Este estado mantiene el robot detenido durante un tiempo definido, permitiendo que
se realicen acciones posteriores o simplemente esperando antes de finalizar la acción.
\index{execute() (método de squad\_approach\_control\_action.WaitState)@\spxentry{execute()}\spxextra{método de squad\_approach\_control\_action.WaitState}}

\begin{fulllineitems}
\phantomsection\label{\detokenize{squad_approach_control_action:squad_approach_control_action.WaitState.execute}}
\pysigstartsignatures
\pysiglinewithargsret{\sphinxbfcode{\sphinxupquote{execute}}}{\sphinxparam{\DUrole{n,n}{userdata}}}{}
\pysigstopsignatures
\sphinxAtStartPar
Ejecuta la lógica del estado de espera.

\sphinxAtStartPar
Mantiene el robot detenido durante el tiempo especificado. Monitorea continuamente
si se ha solicitado una preempción para poder interrumpir la espera si es necesario.
\begin{quote}\begin{description}
\sphinxlineitem{Parámetros}
\sphinxAtStartPar
\sphinxstyleliteralstrong{\sphinxupquote{userdata}} (\sphinxstyleliteralemphasis{\sphinxupquote{smach.UserData}}) \textendash{} Datos de usuario proporcionados por SMACH.

\sphinxlineitem{Devuelve}
\sphinxAtStartPar
Outcome del estado.

\sphinxlineitem{Tipo del valor devuelto}
\sphinxAtStartPar
\sphinxhref{https://docs.python.org/3/library/stdtypes.html\#str}{str}

\end{description}\end{quote}

\end{fulllineitems}


\end{fulllineitems}


\sphinxstepscope


\subsection{Control Autónomo (squad\_autonomous\_control\_action)}
\label{\detokenize{squad_autonomous_control_action:control-autonomo-squad-autonomous-control-action}}\label{\detokenize{squad_autonomous_control_action::doc}}
\sphinxAtStartPar
Este nodo permite la navegación autónoma evitando obstáculos.

\sphinxAtStartPar
\sphinxstylestrong{Descripción del Nodo:}
\begin{itemize}
\item {} 
\sphinxAtStartPar
Segmenta el entorno en regiones (frente, izquierda, derecha, etc.) usando datos de LIDAR.

\item {} 
\sphinxAtStartPar
Toma decisiones de movimiento en función de distancias detectadas.

\item {} 
\sphinxAtStartPar
Configurable con parámetros para velocidades y distancias mínimas.

\end{itemize}

\sphinxAtStartPar
\sphinxstylestrong{Parámetros Importantes:}
\begin{itemize}
\item {} 
\sphinxAtStartPar
\sphinxstylestrong{\textasciigrave{}front\_distance\_threshold\textasciigrave{}}: Distancia mínima para avanzar.

\item {} 
\sphinxAtStartPar
\sphinxstylestrong{\textasciigrave{}side\_distance\_threshold\textasciigrave{}}: Distancia mínima lateral.

\item {} 
\sphinxAtStartPar
\sphinxstylestrong{\textasciigrave{}linear\_speed\textasciigrave{}}: Velocidad lineal del robot.

\end{itemize}
\phantomsection\label{\detokenize{squad_autonomous_control_action:module-squad_autonomous_control_action}}\index{module@\spxentry{module}!squad\_autonomous\_control\_action@\spxentry{squad\_autonomous\_control\_action}}\index{squad\_autonomous\_control\_action@\spxentry{squad\_autonomous\_control\_action}!module@\spxentry{module}}
\sphinxAtStartPar
Servidor de acción para el control autónomo del robot.

\sphinxAtStartPar
Este nodo implementa una acción que permite controlar el robot de manera autónoma,
evitando obstáculos y colisiones con las paredes. La acción puede ser activada o
cancelada mientras se cambia de modo. Está diseñada para ser extendida con algoritmos
de control más avanzados en el futuro.
\index{AutonomousControlActionServer (clase en squad\_autonomous\_control\_action)@\spxentry{AutonomousControlActionServer}\spxextra{clase en squad\_autonomous\_control\_action}}

\begin{fulllineitems}
\phantomsection\label{\detokenize{squad_autonomous_control_action:squad_autonomous_control_action.AutonomousControlActionServer}}
\pysigstartsignatures
\pysiglinewithargsret{\sphinxbfcode{\sphinxupquote{class\DUrole{w,w}{  }}}\sphinxcode{\sphinxupquote{squad\_autonomous\_control\_action.}}\sphinxbfcode{\sphinxupquote{AutonomousControlActionServer}}}{\sphinxparam{\DUrole{n,n}{name}}}{}
\pysigstopsignatures
\sphinxAtStartPar
Bases: \sphinxhref{https://docs.python.org/3/library/functions.html\#object}{\sphinxcode{\sphinxupquote{object}}}

\sphinxAtStartPar
Servidor de acción para el control autónomo del robot.

\sphinxAtStartPar
Esta clase implementa la lógica para controlar el robot de manera autónoma,
evitando colisiones con obstáculos detectados por el sensor LIDAR.
\index{decide\_motion() (método de squad\_autonomous\_control\_action.AutonomousControlActionServer)@\spxentry{decide\_motion()}\spxextra{método de squad\_autonomous\_control\_action.AutonomousControlActionServer}}

\begin{fulllineitems}
\phantomsection\label{\detokenize{squad_autonomous_control_action:squad_autonomous_control_action.AutonomousControlActionServer.decide_motion}}
\pysigstartsignatures
\pysiglinewithargsret{\sphinxbfcode{\sphinxupquote{decide\_motion}}}{\sphinxparam{\DUrole{n,n}{regions}}}{}
\pysigstopsignatures
\sphinxAtStartPar
Decide el movimiento del robot basado en las regiones del LIDAR.
\begin{quote}\begin{description}
\sphinxlineitem{Parámetros}
\sphinxAtStartPar
\sphinxstyleliteralstrong{\sphinxupquote{regions}} \textendash{} Diccionario con las distancias mínimas en cada región.

\sphinxlineitem{Devuelve}
\sphinxAtStartPar
Mensaje Twist con las velocidades lineal y angular.

\end{description}\end{quote}

\end{fulllineitems}

\index{execute\_cb() (método de squad\_autonomous\_control\_action.AutonomousControlActionServer)@\spxentry{execute\_cb()}\spxextra{método de squad\_autonomous\_control\_action.AutonomousControlActionServer}}

\begin{fulllineitems}
\phantomsection\label{\detokenize{squad_autonomous_control_action:squad_autonomous_control_action.AutonomousControlActionServer.execute_cb}}
\pysigstartsignatures
\pysiglinewithargsret{\sphinxbfcode{\sphinxupquote{execute\_cb}}}{\sphinxparam{\DUrole{n,n}{goal}}}{}
\pysigstopsignatures
\sphinxAtStartPar
Método de ejecución de la acción.

\sphinxAtStartPar
Controla el robot de manera autónoma, evitando obstáculos basados en los
datos del LIDAR. La acción puede ser preempted si se recibe una solicitud
de cancelación.
\begin{quote}\begin{description}
\sphinxlineitem{Parámetros}
\sphinxAtStartPar
\sphinxstyleliteralstrong{\sphinxupquote{goal}} \textendash{} Objetivo de la acción (no utilizado en este caso).

\end{description}\end{quote}

\end{fulllineitems}

\index{get\_laser\_regions() (método de squad\_autonomous\_control\_action.AutonomousControlActionServer)@\spxentry{get\_laser\_regions()}\spxextra{método de squad\_autonomous\_control\_action.AutonomousControlActionServer}}

\begin{fulllineitems}
\phantomsection\label{\detokenize{squad_autonomous_control_action:squad_autonomous_control_action.AutonomousControlActionServer.get_laser_regions}}
\pysigstartsignatures
\pysiglinewithargsret{\sphinxbfcode{\sphinxupquote{get\_laser\_regions}}}{}{}
\pysigstopsignatures
\sphinxAtStartPar
Divide las lecturas del LIDAR en regiones.
\begin{quote}\begin{description}
\sphinxlineitem{Devuelve}
\sphinxAtStartPar
Diccionario con las distancias mínimas en cada región.

\end{description}\end{quote}

\end{fulllineitems}

\index{laser\_callback() (método de squad\_autonomous\_control\_action.AutonomousControlActionServer)@\spxentry{laser\_callback()}\spxextra{método de squad\_autonomous\_control\_action.AutonomousControlActionServer}}

\begin{fulllineitems}
\phantomsection\label{\detokenize{squad_autonomous_control_action:squad_autonomous_control_action.AutonomousControlActionServer.laser_callback}}
\pysigstartsignatures
\pysiglinewithargsret{\sphinxbfcode{\sphinxupquote{laser\_callback}}}{\sphinxparam{\DUrole{n,n}{data}}}{}
\pysigstopsignatures
\sphinxAtStartPar
Callback para almacenar los datos del LIDAR.
\begin{quote}\begin{description}
\sphinxlineitem{Parámetros}
\sphinxAtStartPar
\sphinxstyleliteralstrong{\sphinxupquote{data}} \textendash{} Mensaje de tipo LaserScan con los datos del sensor.

\end{description}\end{quote}

\end{fulllineitems}

\index{stop\_robot() (método de squad\_autonomous\_control\_action.AutonomousControlActionServer)@\spxentry{stop\_robot()}\spxextra{método de squad\_autonomous\_control\_action.AutonomousControlActionServer}}

\begin{fulllineitems}
\phantomsection\label{\detokenize{squad_autonomous_control_action:squad_autonomous_control_action.AutonomousControlActionServer.stop_robot}}
\pysigstartsignatures
\pysiglinewithargsret{\sphinxbfcode{\sphinxupquote{stop\_robot}}}{}{}
\pysigstopsignatures
\sphinxAtStartPar
Detiene el robot publicando velocidades cero.

\end{fulllineitems}


\end{fulllineitems}



\chapter{Indices y tablas}
\label{\detokenize{index:indices-y-tablas}}\begin{itemize}
\item {} 
\sphinxAtStartPar
\DUrole{xref,std,std-ref}{genindex}

\end{itemize}


\renewcommand{\indexname}{Índice de Módulos Python}
\begin{sphinxtheindex}
\let\bigletter\sphinxstyleindexlettergroup
\bigletter{s}
\item\relax\sphinxstyleindexentry{squad\_approach\_control\_action}\sphinxstyleindexpageref{squad_approach_control_action:\detokenize{module-squad_approach_control_action}}
\item\relax\sphinxstyleindexentry{squad\_autonomous\_control\_action}\sphinxstyleindexpageref{squad_autonomous_control_action:\detokenize{module-squad_autonomous_control_action}}
\item\relax\sphinxstyleindexentry{squad\_object\_detection\_action}\sphinxstyleindexpageref{squad_object_detection_action:\detokenize{module-squad_object_detection_action}}
\item\relax\sphinxstyleindexentry{squad\_state\_manager}\sphinxstyleindexpageref{squad_state_manager:\detokenize{module-squad_state_manager}}
\end{sphinxtheindex}

\renewcommand{\indexname}{Índice}
\printindex
\end{document}